%%
%% Author: Alexandre Bartel
%%
\documentclass[a4paper, 11pt]{article}
\usepackage[utf8]{inputenc}
\usepackage[dvipsnames]{xcolor}
\usepackage{graphicx}
\usepackage{tikz}
\usepackage{etoolbox}

%% helvetica fonts
\usepackage[scaled]{helvet}
\renewcommand\familydefault{\sfdefault}
\usepackage[T1]{fontenc}
\usepackage{lipsum}

%% line spacing
\renewcommand{\baselinestretch}{1.50}\normalsize

%% spaces between paragraphs, etc..
\usepackage{parskip}
%\setlength{\parindent}{0pt}
\setlength{\parskip}{0em}
%\setlength{\parskip}{0em}
%\setlength{\parindent}{0em}

\usepackage{titlesec}
%% \titlespacing*{<command>}{<left>}{<before-sep>}{<after-sep>}
\titlespacing*{\section}{0pt}{.1pt}{.5pt}
\titlespacing*{\subsection}{0pt}{.1pt}{.5pt}
\titlespacing*{\subsubsection}{0pt}{.1pt}{.5pt}
\titlespacing*{\paragraph}{0pt}{.1pt}{2pt}

\usepackage{hyperref}
\def\UrlBreaks{\do\/\do-}
\def\replef{Bar}
\def\reprig{xan}

\usepackage{lastpage}
\usepackage{fancyhdr}
\cfoot{\thepage\ of \pageref{LastPage}}

%% for tables
\usepackage{multirow}
%\usepackage{pbox}
\usepackage{makecell}
\usepackage{colortbl}

%% margins. footskip = size of footer, bottom = size of bottom incluing footskip!
\usepackage[a4paper,
bottom=20mm, top=15mm, left=15mm, right=15mm,
headheight=5mm,
headsep=10mm, footskip=5mm,
includeheadfoot,
%includehead, includefoot,
]{geometry}
%\addtolength{\oddsidemargin}{-.875in}
%\addtolength{\evensidemargin}{-.875in}
%\addtolength{\textwidth}{1.75in}
%\addtolength{\topmargin}{-.875in}
%\addtolength{\textheight}{1.75in}

%% headers / footers
\usepackage{fancyhdr}
\pagestyle{fancy}
\fancyhf{}
\renewcommand{\headrulewidth}{0pt}
\rhead{\includegraphics[height=1cm]{figures/header-right}}
\lhead{\includegraphics[height=1cm]{figures/header-left}}
\chead{}%\textcolor{lightgray}{\thepage}}
\rfoot{\includegraphics[height=.5cm]{figures/footer-right}}
\lfoot{\includegraphics[height=.5cm]{figures/footer-left}}
\appto\replef{tel}
\appto\reprig{dre}
\preto\reprig{Ale}
\cfoot{\scriptsize {\tikz{ \path (0,0) node[color=black!0.5] {\replef{} yalishan}}}%
FNR / B.P. 1777 / L-1017 Luxembourg / T +352 26 19 25 1 / F +352 26 19 25 35 / www.fnr.lu %
\tikz{ \path (0,0) node[color=black!0.5] {\reprig{} da}}}%
\fancypagestyle{plain}{\pagestyle{fancy}} %% add header/footer also on the first page

%% space before title
\usepackage{titling}
%\setlength{\droptitle}{-4em}     % Eliminate the default vertical space
\addtolength{\droptitle}{4cm}   % Only a guess. Use this for adjustment

%opening
\title{\bf \textcolor{Plum}{Project Description Form} \\ \textcolor{Gray}{Core 20XX Call}}
\author{\vspace{-5ex}}
\date{\vspace{-5ex}}

\usepackage{natbib}

% Please carefully read the Guidelines for Applicants before starting the description of your research proposal.
% Bear in mind that the proposal will be evaluated according to the selection criteria set out in the guidelines
% for applicants and in the peer-review guidelines. To be successful, the description has to clearly address these criteria.
% The font type to be used by default is Arial. If the document preparation system you use does not have Arial,
% chose a font type that is equivalent to Arial in terms of space usage (e.g. Helvetica for LaTeX). Independent of
% the document preparation system, the page size to use is A4, all margins (top, bottom, left, right)
% must be at least 15 mm (not including any footers or headers), the minimum font size allowed is 11 points and
% the line spacing is minimum 1.5.
% The maximum number of pages indicated for each section/heading must be respected.
% The Project description cannot be submitted alone. Before uploading the document to the online application form,
% it has to be converted to .pdf
% PROJECT DESCRIPTION
%     1. Description of the Proposed Research Project. (max. 7 pages for 1.1. - 1.4.)
%         1.1 Introduction
%         1.2 Relevant state-of-the art and your own contribution to it
%         1.3 Hypotheses, project objectives and contribution to knowledge development in the research field
%         1.4 Methods and approach
%         1.5 Ethical considerations (if applicable, max. 2 pages)
%     2. Project plan (3 to 10 pages)
%     3. Risk management and quality assurance (max. 1 page)
%     4. Project Outputs
%      4.1 Impact of research results (max 2. pages)
%      4.2 PhD student supervision and research lines (if applicable, 1 page/PhD candidate)
%      4.3 In addition, for CORE Junior Track: Advancement of the Junior PI’s research career (max. 2 pages)
%     5. Project Participants and Management
%      5.1 Description of the consortium, communication and decision-making (max. 1 page)
%      5.2 Summaries (term sheets) of the Consortium agreement and/or the Intellectual Property Rights (IPR) agreement (max 1 page)
%      5.3 Track record of the PI and applicant team (competence in the domain, publications, past fundings as PI) (max. 2 pages)
%     6. Comments on Resubmission (if applicable, max. 1 page)
%     7. Bibliography / References (max. 3 pages)
\begin{document}

\vspace{10cm}
\maketitle

\begin{center}
\begin{tabular}{|p{4.5cm}|p{0.6\textwidth}|}
\hline
\bf Project Acronym  &  \\ \hline
\bf Principal Investigator (PI)  &  Dr. Alexandre Bartel \\ \hline
\bf Host Institution  & \\ \hline
\end{tabular}
\end{center}

\newpage
\section{Description of the Proposed Research Project (max 7 pages 1.1 - 1.4)}\label{sec:description}
\subsection{Introduction}\label{sec:introduction}
\paragraph{P1}
\lipsum[1]
Non d'une pipe~\cite{bartel2012dexpler}!

\paragraph{P2}
\lipsum[2]

\paragraph{P3}
\lipsum[3]

\subsection{Relevant state-of-the art and your own contribution to it}\label{sec:stateoftheart}
\subsubsection{sub sub section 1}
\paragraph{P1.}
\lipsum[7]

\subsubsection{sub sub section 2}
\lipsum[8]

\subsection{Hypotheses, project objectives and contribution to knowledge development in the research field}
\paragraph{P1}
\lipsum[4]

\paragraph{P2}
\lipsum[5]

\paragraph{P3}
\lipsum[6]

\subsection{Methods and approach}\label{sec:approach}

\subsection{Ethical considerations (if applicable, max. 2 pages)}
N/A

%%%
%%%
\newpage
\section{Project Plan (3 to 10 pages)}\label{sec:plan}
\renewcommand{\arraystretch}{0.75}
\begin{center}
\begin{tabular}{|p{.22\textwidth}|p{.22\textwidth}|p{.22\textwidth}|p{.22\textwidth}|}
\hline
 \bf WP Number & \multicolumn{3}{|l|}{} \\ \hline
 \bf WP title  & \multicolumn{3}{|l|}{} \\ \hline
 \bf WP leader & \multicolumn{3}{|l|}{} \\ \hline
 \bf Start date& T     & \bf End date    & T \\ \hline
 \multicolumn{4}{|l|}{\bf Objective} \\ \hline
 \multicolumn{4}{|p{.99\textwidth}|}{
 } \\ \hline
 \multicolumn{4}{|l|}{\bf Tasks} \\ \hline
 \multicolumn{4}{|p{.99\textwidth}|}{
 } \\ \hline
 \multicolumn{4}{|l|}{\bf Interdependence with other work packages} \\ \hline
 \multicolumn{4}{|l|}{
 } \\ \hline
 \multicolumn{4}{|l|}{\bf Deliverables and milestones} \\ \hline
 \multicolumn{4}{|p{.95\textwidth}|}{
 } \\ \hline
 \multicolumn{4}{|l|}{\bf Human resources} \\ \hline
 \bf Name of researcher & \bf Partner     & \bf Qualification Level & \bf Person*months \\ \hline
\end{tabular}
\end{center}

%%%
%%%
\newpage
\section{Risk management and quality assurance (max. 1 page)}

\begin{center}
{
\renewcommand{\arraystretch}{0.7}
 \begin{tabular}{|p{.016\textwidth}|p{.16\textwidth}|p{.11\textwidth}|p{.07\textwidth}|p{.24\textwidth}|p{.22\textwidth}|}
 \hline
  \rowcolor{gray!20} \multicolumn{4}{|c|}{\bf Risk } & \multicolumn{2}{|c|}{\bf Mitigation or Contingency actions } \\ \hline
  \rowcolor{gray!20} \bf N$^o$ & \bf Identified Risk & \bf Likelihood & \bf Impact & \bf Action & \bf Impact of Action \\ \hline
  %%
  \rowcolor{gray!20} \multicolumn{6}{|c|}{\bf Technical Risks } \\ \hline
  %%
  \rowcolor{gray!20} \multicolumn{6}{|c|}{\bf Non Technical Risks } \\ \hline
  %%
  \end{tabular}
}
\end{center}

%%%
%%%
\newpage
\section{Project Outputs}
\subsection{Impact of research results (max 2. pages)}

\subsection{PhD student supervision and research lines (max. 1 page/PhD candidate)}

\subsection{Advancement of the Junior PI’s research career (max. 2 pages)}

%%%
%%%
\newpage
\section{Project Participants and Management}
\subsection{Description of the consortium, communication and decision-making (max. 1 page)}
\subsection{Summaries (term sheets) of the Consortium agreement and/or the Intellectual Property Rights (IPR) agreement (max 1 page)}
\subsection{Track record of the PI and applicant team (competence in the domain, publications, past fundings as PI) (max. 2 pages)}

%%%
%%%
\newpage
\section{Comments on Resubmission (only if applicable, max. 1 page)}

%%%
%%%
\newpage
\bibliographystyle{plain}
@inproceedings{bartel2012dexpler,
  title={{\bf Dexpler: converting android dalvik bytecode to jimple for static analysis with soot}},
  author={{\bf Bartel, Alexandre and Klein, Jacques and Le Traon, Yves and Monperrus, Martin}},
  booktitle={{\bf Proceedings of the ACM SIGPLAN International Workshop on State of the Art in Java Program analysis}},
  year={{\bf 2012}},
  organization={{\bf ACM}}
}

\end{document}