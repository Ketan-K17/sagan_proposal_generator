
\documentclass[a4paper,12pt]{article}
\usepackage{graphicx}
\usepackage{amsmath}
\usepackage{hyperref}
\usepackage{geometry}
\geometry{margin=1in}

\title{AI-Driven Systems for Autonomous Spacecraft Operations}
\author{}
\date{}

\begin{document}
\maketitle
\begin{abstract}
The AI-Driven Systems for Autonomous Spacecraft Operations project explores the transformative role of artificial intelligence (AI) in enhancing the efficiency, autonomy, and capabilities of space missions. By leveraging machine learning algorithms for real-time data analysis, AI facilitates advancements in Earth observation, navigation, and satellite operations, reducing reliance on intensive testing and lowering costs. The project addresses the challenges of integrating AI into spacecraft, emphasizing the need for robust, fault-tolerant systems that can withstand the harsh conditions of space. A dedicated AI design, verification, and certification framework is proposed to ensure system reliability and resilience. The research highlights AI's potential in mission planning, anomaly detection, and autonomous decision-making, enabling spacecraft to adapt to unforeseen situations and operate with minimal human intervention. The project underscores the importance of human-on-the-loop control, balancing machine autonomy with human oversight to enhance mission success. As AI continues to evolve, it promises to revolutionize spacecraft design, optimize mission objectives, and facilitate ambitious exploration endeavors, while addressing technical, ethical, and legal challenges to ensure safe and secure operations.
\end{abstract}
\tableofcontents
\newpage
\section{Introduction: Originality of the Research Project}

Source
DB
DB
Query
Object
Graph
Fig. 1
Functional diagram of the Prepare step of the AI/ML workflow.
framework. A number of FOSS and COTS solutions exist and are implemented either as tools or APIs at the project
level (e.g., DVC∗or the tf.data.Dataset module†) or as an enterprise-level central dataset repository (e.g., Collibra‡).
Beyond storing the data, centralized dataset repositories provide additional features that aid in the management of

work on systems capable of generating reliable responses to these parameters. Furthermore,
it is necessary to identify tools to improve access to information in the mission-design and
planning phases to manage the information necessary for its success. These tools are called
design engineering assistants and can support decision-making for complex engineering
problems such as initial input estimation, assisting experts by answering queries related to

trajectories. However, the true nature of most design problems is multi-objective,
potentially also including integer decisions variables or non-linear constraints. In
this multi-objective setting, the concept of the best design (i.e. the global optimum)
is substituted with that of a Pareto front, a collection of non-dominated solutions
expressing the trade-offs between different conflicting objectives. Consequently, a
set of best possible solutions (Pareto-optimal front) is required to guide engineering

spacecraft design is a vast effort, branching out into many fields of science and 
engineering. The proposed research obtains several important results towards the 
design of space missions that provide higher utility to the stakeholders, by being 
more optimized and not bound to the stagnancy of conservative mission design 
approaches. These improvements are obtained through innovations in three aspects 
of the mission design: 
• exploring the alternative concepts thoroughly and more efficiently

Explainable
• Model uncertainty
• Model architecture/complexity
• Explainability
• Explainable AI techniques
• Continuous monitoring
• Root cause analysis
User Experience
Rapid
• Development time
• Ease of update
• Model/Data repository
• COTS development tools/services
High Value
• Non-recurring cost
• Recurring cost
• Improvement over other methods
• Automated pipelines
• Reuse common solutions
• Reduce duplication of efforts
Easy to use
• UX/UI intuitiveness
• Ease of integration
• Standards conformity

occur in any stage of the mission.
Updates, for example,
can come from downlink data, both at the tactical level
(e.g. a magnetosphere model might need adjustments), or
at the strategic level with trend and history analysis (e.g.,
the duration of plumes at a certain location can be better
estimated given historical data from previous flybys).
Figure 7. Variability Definition: allows scientists,
engineers, autonomy engineers and operators to input
uncertainty with respect to a multitude of aspects that might

of the workflow. Additionally, high-quality datasets for these environments can be difficult to come by and relying on
augmentation or simulations potentially introduces unknown vulnerabilities. Questions, such as "How will the model
respond to novel inputs?" or "Under what conditions do we expect the outputs of the model to be valid?" need to be
addressed. The performance of an AI/ML system will naturally degrade over time whether due to data drift, changes in

can be determined by performance of the model and/or a schedule.
After development, the artifacts are tested to ensure proper function and performance. Next a final review is conducted
before the packaged automated model pipeline is released for delivery to the target environment.
The major steps in the Develop phase are largely the same across the three environments and the differences at
workflow level have been discussed in the previous sections. The main differences are the types of artifacts built during

across those meetings, though we also encouraged them to
break into reflective discussions about the process and tools
along the way. We also collected feedback in the form of
survey responses in a journal they used for each day. We
had implemented the tools as click-through prototypes only,
and user study facilitator needed to “drive” the tools at the the
operator’s request. This dynamic evoked dialogue about what
information they needed to see and why.
14

aspects, but, among all, is mainly described by the network architecture and
the state-action spaces; on the other side, the reward formulation embeds the
mission objectives. Once these two, or more if needed, cases have been selected,
the workflow proposes to train and test them with different levels of initial
conditions. This difference can be interpreted in distinct ways: for instance, in
Section 4.3, it concerns the randomness level of the initial conditions; while, in

innovators. We will create opportunities to crossflow 
high caliber personnel from industry into government 
positions and USSF members to industry or other 
government agencies to remain current in technology 
and best practices.  
 
REALISTIC TRAINING 
 
We will make every effort to train in realistic, contested 
conditions. Our forces are always in competition, and our 
capabilities are likely among the first targets of an 
aggressor’s action. In both competition and conflict

other similar details;  
• Using probabilistic, rather than deterministic, transition rules. 
Population-based algorithms adopt a similar approach, regardless of 
the applied paradigm and follow from the algorithm below.  
1. Initialise the population;  
2. Fitness is calculated for each individual in the population;  
3. Produce a new population-based on some rules that strictly depend 
on the fitness of each individual;  
4. Repeat steps 2–4 until a condition is met. 
4.1.2. Machine learning techniques

6.1.
Overview of the Methodology
these four coupled models are trained with two different types of random initial
conditions following the strategy outlined in the Fig. 6.1. In this way, it was
possible to understand how the agent is robust and adaptable. In Section 4.4
the same kind of analysis has been followed to increase again the complexity
of the model, in this case in terms of state and action space.
Figure 6.1: Schematization of the proposed workflow methodology.
141

bold, future focus informs our force design. We will 
deliver a streamlined, agile, and innovative organization 
that sets a new standard in the Department of Defense. 
 
During this period of transformation, our forces must 
continue to deliver the effects our Nation and Joint Force 
count on without fail. Commanders responsible for those 
missions will prioritize efforts to ensure they continue 
seamlessly despite the disruptions inevitable during

standing, should be investigated first. Classical and specialised methods, 
on the other hand, are often naive, whereas heuristic and metaheuristic 
paradigms can be utilised to various conditions. One key advantage of 
heuristic and metaheuristic paradigms is their robustness. In this 
context, robustness refers to an algorithm’s ability to solve a wide range 
of problems and even multiple sorts of problems, with only slight 
changes to account for each problem’s specific properties. A stochastic

to the reader. We decided to give importance to work published in the last few years,
avoiding the historical perspective of older and well-established fundamental works.
Additionally, we decided to avoid publications which are strongly speculative in
nature: while visionary ideas are interesting to follow, an important requirement for
this survey was a well-motivated applicability for a space-related challenge, ideally
inspired by an already established or newly proposed mission concept. This narrow

that the 35% of the new generation is formed by individuals picked from the old 
generation. The selected value ensures a balance between effectiveness of the 
search (lower elite population fractions) and survival of fit individuals (higher elite 
population fractions). 
8.5 Results 
The investigation on methodologies to improve and automate the space mission and 
spacecraft design is a vast effort, branching out into many fields of science and

challenges and unlocking new opportunities. By 
leveraging AI techniques, scientists and engineers 
can enhance data analysis, enable autonomous 
systems, improve navigation, and achieve greater 
efficiency in space missions. 
 
Methodology: 
To conduct this research initially, an extensive 
literature search was performed using reputable 
scientific databases, including IEEE Xplore, ACM 
Digital Library, and Google Scholar. The search 
keywords included "artificial intelligence," "AI,"

of the main research topic, corroborating its procedures and techniques. All
three cases confirmed the good quality of the proposed methodology, achieving
satisfying performance levels in each of the applications’ frameworks.
Working for a Ph.D. is not an easy road, it is a path full of stops, sudden turns,
backs out and comebacks, and surely does not exhaustively explore a research
field. But certainly, it helps pushing towards the accomplishment of evermore

spacecraft design is a vast effort, branching out into many fields of science and 
engineering. The proposed research obtains several important results towards the 
design of space missions that provide higher utility to the stakeholders, by being 
more optimized and not bound to the stagnancy of conservative mission design 
approaches. These improvements are obtained through innovations in three aspects 
of the mission design: 
• exploring the alternative concepts thoroughly and more efficiently

to the projected mission progress and performance. Itera-
tions continue, as the team decides to make minor tweaks
or drop problematic goals entirely.
The team can inspect
individual cases or “clusters” of related cases to understand
outcomes that might happen onboard and investigate prob-
lematic plans that approach undesired limits. Explanation of
these problematic cases (either by manually inspecting logs,
state history, timelines and traces or by using an automated

Figure 10. Mission Impact tool: shows an overview of the
simulations and the impact of updated or newly added goals
to the mission progress
actual telemetry from downlink and compare it to modeled
predicted data (predicts) to support the analysis of onboard
health and safety and anomaly detection[16]. The tool resem-
bles conventional downlink analysis tools, plotting onboard
state over time overlaid with events, and a list of EVRs. The
predicts are clustered, since the uplink process for autonomy

Progress in Aerospace Sciences 144 (2024) 100960
27
developed by The Aerospace Corporation (Fig. 30) looks at a system’s 
capacity to fulfil mission objectives throughout the course of its full 
lifecycle. Once the mission’s requirements and specific functions have 
been determined, the possible threats that can affect the mission should 
be identified, along with potential strategies. The threat’s goals must be 
also identified because the strategy is a tool for attaining them [322].

experienced autonomy. In particular, we chose to focus on
how scientists would react to probabilistic resource conflicts,
whether operators would trust and accept a non-deterministic
uplink plan, and whether operators would feel confident in
their retrospective reconstruction of onboard behavior and
safety upon downlink.
To study these particular facets of operations, we selected the
previously described “Mapping Triton and Plume Detection”
scenario (see Section 2), and elaborated preliminary tools and

to 70%.
The impacted scientist felt that the changes had
compromised their original observation even thought it was
still likely to occur, and, as a result, the participants discussed
several strategies to make a compromise or otherwise relax
the plan without using autonomy.
The impacted scientist
explained “I would have pushed harder if there was any
indication that [the baseline plumes] or both were super
important and we were not able to image them again.” At the

\section{Hypothesis, Research Objectives and Envisaged Methodology}

Machines, Ensemble Learning. 
Several other methods consider prior knowledge during learning: in this case, 
the effects of knowledge representation and learning are joined together. Current-
best-hypothesis search and Least-commitment search are examples of these 
algorithms. Efforts are also being spent in developing methodologies to extract 
general knowledge from specific examples. Several different types of learning have 
been developed, including Explanation-based learning, Relevance-based learning,

paramount importance are also the 
operation research and the Markov 
decision 
processes. 
Some of the fundamental questions of 
the 
field, 
related 
to 
AI: 
- how should we make decisions to 
improve 
the 
outcomes? 
- how can we change these decisions 
when the outcomes are evaluated in the 
far future, or when boundary conditions 
vary? 
Neuroscience 
Neuroscience is involved with 
studying the brain, which is the main 
element of the nervous systems in 
human beings. Despite the majority of

Analysis
Once we concluded the study, we transcribed the audio from
the sessions and compiled the survey responses. To identify
themes, we grouped clusters of related observations into
affinity groups, focusing on topics that related to our research
questions.
Findings
In this section, we describe preliminary themes that emerged
from our analysis.
The user study participants in general
successfully used the tools and series of steps to complete
their operations tasks. The operators’ participation, inquiries,

version control that tracks the original source and any modifications to a dataset, ensuring traceability of the data lineage.
The first step in most machine learning projects, once data is in hand, is exploratory data analysis (EDA). The goals
of EDA are to understand the contents of the dataset, identify any trends that might be exploitable via AI/ML, and begin
to shape the hypotheses and approaches to explore. The EDA step is often dataset-specific and requires a knowledgeable

of the workflow. Additionally, high-quality datasets for these environments can be difficult to come by and relying on
augmentation or simulations potentially introduces unknown vulnerabilities. Questions, such as "How will the model
respond to novel inputs?" or "Under what conditions do we expect the outputs of the model to be valid?" need to be
addressed. The performance of an AI/ML system will naturally degrade over time whether due to data drift, changes in

Appl. Sci. 2022, 12, 5106
4 of 21
Therefore, all efforts lead to two common goals: (i) to push toward space exploration
and scientific discoveries; (ii) to improve life on Earth. Human beings play an essential
role in all this, such as specialists who work on the objectives of space missions, astronauts
who experience challenging conditions in space, and ordinary citizens involved in the
achievement of these challenges, particularly those of improving life on Earth.

such observation as a goal in its corresponding campaign.
The tool also allows the direct specification of goals in the
form a set of desired activities (e.g. observation, detection)
without necessarily using the search mechanism. Figure 5
shows an example of the creation of two goals (one con-
ditional on execution of the other) to monitor a particular
location on the surface of Triton, and perform a follow-on
observation if a plume is detected. The figure also shows

plan specification, from strategic to tactical, which largely
aligns with current mission practices of large ground planning
teams at NASA JPL. Building on previous work at JPL [11]
[12] [13] [14] we designed a set of UI tools to progressively
capture and specify intent as science campaigns. In this work
campaigns are composed by: a constrained set of goals (a de-
sired state value or a high level activity, e.g. survey the mag-
netosphere, or monitor for plume activity), metrics to evaluate

Progress in Aerospace Sciences 144 (2024) 100960
27
developed by The Aerospace Corporation (Fig. 30) looks at a system’s 
capacity to fulfil mission objectives throughout the course of its full 
lifecycle. Once the mission’s requirements and specific functions have 
been determined, the possible threats that can affect the mission should 
be identified, along with potential strategies. The threat’s goals must be 
also identified because the strategy is a tool for attaining them [322].

overall public and pre-accession interest in space. We will 
not passively wait for the best to respond to marketing, 
but actively use merit and diversity-based criteria to seek 
the talent we want. 
 
Our efforts to manage, develop, and retain this talent 
will be central to the long-term viability and success of 
the Service. Our small, flat organization allows for 
deliberate individualized development focused on 
building 
space 
warfighters 
with 
the 
necessary

of the main research topic, corroborating its procedures and techniques. All
three cases confirmed the good quality of the proposed methodology, achieving
satisfying performance levels in each of the applications’ frameworks.
Working for a Ph.D. is not an easy road, it is a path full of stops, sudden turns,
backs out and comebacks, and surely does not exhaustively explore a research
field. But certainly, it helps pushing towards the accomplishment of evermore

Analysis
Once we concluded the study, we transcribed the audio from
the sessions and compiled the survey responses. To identify
themes, we grouped clusters of related observations into
affinity groups, focusing on topics that related to our research
questions.
Findings
In this section, we describe preliminary themes that emerged
from our analysis.
The user study participants in general
successfully used the tools and series of steps to complete
their operations tasks. The operators’ participation, inquiries,

challenges and unlocking new opportunities. By 
leveraging AI techniques, scientists and engineers 
can enhance data analysis, enable autonomous 
systems, improve navigation, and achieve greater 
efficiency in space missions. 
 
Methodology: 
To conduct this research initially, an extensive 
literature search was performed using reputable 
scientific databases, including IEEE Xplore, ACM 
Digital Library, and Google Scholar. The search 
keywords included "artificial intelligence," "AI,"

4. Repeat steps 2–4 until a condition is met. 
4.1.2. Machine learning techniques 
ML approaches are a subset of AI techniques that allows for the 
creation of analytical models to be automated. It is a branch of AI based 
on the idea that computers can learn through data, identify patterns and 
make judgments with small or no human intervention. A ML process is 
shown in Fig. 20. A model that can be queried by an application is 
trained based on a data or knowledge base. Regardless of suitable con­

decision-making[31]. Sinha R., (2018), Data Mining techniques can extract valuable insights from this data, such as identifying patterns, 
anomalies, or potential discoveries [32]. Sinha R., (2019), Data Warehousing can provide a centralized repository for storing and 
integrating data from different sources, facilitating comprehensive analysis and knowledge discovery. By combining these technologies

the training data, hoping it would accurately work for the real ones.
Some kind of assurance is needed, that your model has got most of
the patterns from the data correct, and its not picking up too much
on the noise, i.e. overfitting the training data. The process of deciding
whether the numerical results quantifying hypothesized relationships
between variables are acceptable as descriptions of the data, is known
as validation. Generally, an error estimation for the model is made af-

the mission’s objectives.
• Initial conditions robustness: where all the parameters involved in
the random selection and their relative randomness level are defined or
designed.
The adopted workflow based on these three aspects is schematized in Fig. 6.1.
This methodology is proposed as a sort of vademecum or guidelines to follow
for the implementation and testing of a robust DRL-based path or strategy
planning agent. Indeed, all three scenarios that will be analysed in the next

and Lin [19] prove in their study how this approach generally yields
the best performance when compared to other schemes.
2.3
training & validation
A simple,
introductory article
on the topic can be
found on [1]
When building a machine learning model, it is absolutely necessary
to assess the stability and performance of said model over unseen
data, before releasing it as the final product. It cannot just be fit to
the training data, hoping it would accurately work for the real ones.

standing, should be investigated first. Classical and specialised methods, 
on the other hand, are often naive, whereas heuristic and metaheuristic 
paradigms can be utilised to various conditions. One key advantage of 
heuristic and metaheuristic paradigms is their robustness. In this 
context, robustness refers to an algorithm’s ability to solve a wide range 
of problems and even multiple sorts of problems, with only slight 
changes to account for each problem’s specific properties. A stochastic

4. Repeat steps 2–4 until a condition is met. 
4.1.2. Machine learning techniques 
ML approaches are a subset of AI techniques that allows for the 
creation of analytical models to be automated. It is a branch of AI based 
on the idea that computers can learn through data, identify patterns and 
make judgments with small or no human intervention. A ML process is 
shown in Fig. 20. A model that can be queried by an application is 
trained based on a data or knowledge base. Regardless of suitable con­

Hubble Space Telescope and the Kepler mission. Once the schedule is defined, it is 
uploaded to the spacecraft and executed in a time-tagged way. In general, the 
definition of the activities is performed not only for the nominal path, but alternate 
branches of off-nominal conditions are also foreseen and generated. Interestingly, 
the definition of the timeline of operations is a process as time-dependent as the 
execution of the operations itself: in certain cases, the look-ahead period can reach

can be determined by performance of the model and/or a schedule.
After development, the artifacts are tested to ensure proper function and performance. Next a final review is conducted
before the packaged automated model pipeline is released for delivery to the target environment.
The major steps in the Develop phase are largely the same across the three environments and the differences at
workflow level have been discussed in the previous sections. The main differences are the types of artifacts built during

exploits a Deep Autoencoder to perform image compression. 
2.3. Mission Timeline 
The Φsat-2 spacecraft is designed for a 14-month lifetime from launch with potential extension 
up to 2 more years. Figure 5 shows the nominal top-level timeline of the mission, and Table 4 
Summary of Mission PhasesTable 4 provides a description of the activities undertaken during 
each phase. 
 
Figure 5  Mission Timeline 
Phase 
Duration 
Activities 
Launch and Early 
2 weeks 
● Launch and separation

Progress in Aerospace Sciences 144 (2024) 100960
5
c) Life Cycle Parameters: Total time required for operation, budget, as 
well as maximum time required for initial deployment. 
A high-level process for calculating the decentralisation values from 
M3 to M4 is depicted in Fig. 5. Because of the underlying network 
structure, designers are recommended to rely on multi-agent techniques 
that blend system dynamics and evolution with autonomous behaviour 
[34]. 
2.3. DSS classification

to the projected mission progress and performance. Itera-
tions continue, as the team decides to make minor tweaks
or drop problematic goals entirely.
The team can inspect
individual cases or “clusters” of related cases to understand
outcomes that might happen onboard and investigate prob-
lematic plans that approach undesired limits. Explanation of
these problematic cases (either by manually inspecting logs,
state history, timelines and traces or by using an automated

\section{Expected Outcomes / Impact}

changes into the main plan.
Then the team collectively
reviews these preliminary “low fidelity” outputs, implements
iterations as needed and approves advancing to the “high
fidelity” evaluation of possible outcomes and autonomy out-
put.
Such high fidelity evaluation is called here outcome
prediction phase.
At the outcome prediction phase, simulations produce a more
realistic and comprehensive view of the new plan’s impact
to the projected mission progress and performance. Itera-

as the metrics and variability specifications. Figure 9 shows
the predicted outcomes (for the target tasknet) on the left-
hand side, ordered from most likely to least likely, aiding the
operator in more easily deciphering the expected behavior of
the constructed plans. The green and red arrows inform the
impact of an added goal (in this case, observing Plume X)
on the outcome distribution. For example, the percentage of
cases in which Observation A and B will be both performed

and battery status. Since this view displays an aggregate of
all outcomes, the charts on the timeline view showcase the
overlaid results in a gradient-like pattern indicating all the
possible values for the various outcomes. All in all, this view
is crucial for the operator in examining all of the possible
outcomes after running through the prediction engine.
Figure 9. Mission Planning Prediction Results tool: shows
the aggregated summary of all simulation runs for a given
task network

of the outcome of the plan, while higher fidelity predictions,
which we made available later in the activity, provided es-
timations based on a Monte Carlo Simulation approach that
modeled the outcomes across 10,000 scenarios/outcomes.
While participants generally accepted the progression from
low fidelity to high fidelity data, conversations with scientist
participants revealed that some wanted higher fidelity data
available at the beginning of negotiations. In this particular

prognostics system, enabling further predictions. This would enable operators to
avoid problems in advance, reducing the rail delays across their network. This has
been deployed in early stages and is showing promising results in reducing delays.
This has enabled Deutsche Bahn to implement a prototype semi-autonomous
FDIR system.
3.2
Prognostics & Prediction
With the advent of low cost, high volume Internet of Things (IoT) devices,
automated machine health monitoring has become more practical [14]. Machine

inspired by an already established or newly proposed mission concept. This narrow
scope allows our survey to be concise while remaining relevant for the interested
practitioner.
Many times, results obtained by one AI technology for a specific task appear
stunning, but perform rather poorly when transferred to a different task, which often
happens when its strength and weaknesses are not thoroughly understood. However,
due to the pioneering works of many researchers combined with the results of large

paramount importance are also the 
operation research and the Markov 
decision 
processes. 
Some of the fundamental questions of 
the 
field, 
related 
to 
AI: 
- how should we make decisions to 
improve 
the 
outcomes? 
- how can we change these decisions 
when the outcomes are evaluated in the 
far future, or when boundary conditions 
vary? 
Neuroscience 
Neuroscience is involved with 
studying the brain, which is the main 
element of the nervous systems in 
human beings. Despite the majority of

address 
the 
challenges 
and 
opportunities in integrating AI in the field of space 
exploration. 
Furthermore, the selected papers were scrutinized to 
extract relevant data and insights that could 
contribute to the research paper. This involved 
identifying statistical information, experimental 
results, case studies, and any other data that 
supported the findings and conclusions of the 
respective studies. The extracted data and insights 
were then used to strengthen the research paper's

of the main research topic, corroborating its procedures and techniques. All
three cases confirmed the good quality of the proposed methodology, achieving
satisfying performance levels in each of the applications’ frameworks.
Working for a Ph.D. is not an easy road, it is a path full of stops, sudden turns,
backs out and comebacks, and surely does not exhaustively explore a research
field. But certainly, it helps pushing towards the accomplishment of evermore

is more exciting because AI and space technologies offer a wide range of 
opportunities. Nevertheless, the need to understand the ultimate 
outcome of the technology remains unanswered. Not to mention that 
even the scientific research community are unable to agree on a prece­
dent arising from the use of AI. Prominent scientists and industry leaders 
argued that AI could radically transform the way we live and work, 
potentially threating our civilisation and even human survival [343]. A

address 
the 
challenges 
and 
opportunities in integrating AI in the field of space 
exploration. 
Furthermore, the selected papers were scrutinized to 
extract relevant data and insights that could 
contribute to the research paper. This involved 
identifying statistical information, experimental 
results, case studies, and any other data that 
supported the findings and conclusions of the 
respective studies. The extracted data and insights 
were then used to strengthen the research paper's

Dynamics, DARPA). Excellent examples of applications are also to be found in 
interplanetary robotics systems, such as NASA Mars Science Laboratory [55]. 
Robotics – Research applications have differentiated into various fields, 
encompassing aerial, terrestrial and underwater robots: examples are found in the 
heavy industry, in paralyzed people aids, computer vision and so on. Other 
applications involved are the Touring Problems, VLSI layouts, Automatic 
Assembly Sequencing and so on.

challenges and unlocking new opportunities. By 
leveraging AI techniques, scientists and engineers 
can enhance data analysis, enable autonomous 
systems, improve navigation, and achieve greater 
efficiency in space missions. 
 
Methodology: 
To conduct this research initially, an extensive 
literature search was performed using reputable 
scientific databases, including IEEE Xplore, ACM 
Digital Library, and Google Scholar. The search 
keywords included "artificial intelligence," "AI,"

showcasing 
its 
applications 
and 
benefits. 
Researchers have explored the use of AI in satellite 
operations, enabling efficient monitoring, control, 
and maintenance of satellites in orbit. Moreover, AI 
techniques have been employed for data analysis 
from space missions, enabling the extraction of 
valuable 
insights 
and 
facilitating 
scientific 
discoveries. Robotics is another area where AI has 
demonstrated its potential, with autonomous robots 
being deployed for tasks such as planetary

inspired by an already established or newly proposed mission concept. This narrow
scope allows our survey to be concise while remaining relevant for the interested
practitioner.
Many times, results obtained by one AI technology for a specific task appear
stunning, but perform rather poorly when transferred to a different task, which often
happens when its strength and weaknesses are not thoroughly understood. However,
due to the pioneering works of many researchers combined with the results of large

plans to facilitate an iterative design process of science in-
tent, including capturing intent and constructing plans with
that intent.
We focus on a workflow that includes intent
capture/modeling, outcome/execution prediction, explanation
of elements in the predicted outcomes (e.g.
undesirable
performance), as well as advisory techniques (e.g., “to fix
undesirable behavior, add/change this constraint”).
The
proposed workflow aims to facilitate the operators’ learning

of the outcome of the plan, while higher fidelity predictions,
which we made available later in the activity, provided es-
timations based on a Monte Carlo Simulation approach that
modeled the outcomes across 10,000 scenarios/outcomes.
While participants generally accepted the progression from
low fidelity to high fidelity data, conversations with scientist
participants revealed that some wanted higher fidelity data
available at the beginning of negotiations. In this particular

plan specification, from strategic to tactical, which largely
aligns with current mission practices of large ground planning
teams at NASA JPL. Building on previous work at JPL [11]
[12] [13] [14] we designed a set of UI tools to progressively
capture and specify intent as science campaigns. In this work
campaigns are composed by: a constrained set of goals (a de-
sired state value or a high level activity, e.g. survey the mag-
netosphere, or monitor for plume activity), metrics to evaluate

liminary design hadn’t anticipated. For example, we had de-
signed the system to include progressive disclosure of higher
fidelity science goal prediction details.
These predictions
indicated the probability of each goal being successfully
executed, given the order of operations and science targets.
Lower-fidelity predictions, which we made available to par-
ticipants at the activity start, gave operators a rough estimate
of the outcome of the plan, while higher fidelity predictions,

intent, and updates that align evolving requirements.  We 
value clear verbal and written communication oriented to 
inform decisions and implement actions. 
We also value design approaches to critical thinking, and 
data-driven problem solving. Over the next year, the 
Director of Staff will publish guidance that establishes a 
standard 
for 
how 
space 
professionals 
approach 
structured data-driven decision-making. Like MTOs, a 
standardized process is not intended to constrain thinking

metric specification, showing the number of plumes metric
as an example. Progress and impact is also shown for each
campaign, based on the inputs given in the Mission Planning
tool.
Metrics are key inputs to evaluate both current and
predicted spacecraft performance.
Note that most of the
metrics (if not all of them) are usually captured and specified
early in the mission, or at the strategic planning level.
Figure 6. Metric Definition: allows scientists, engineers,

of the workflow. Additionally, high-quality datasets for these environments can be difficult to come by and relying on
augmentation or simulations potentially introduces unknown vulnerabilities. Questions, such as "How will the model
respond to novel inputs?" or "Under what conditions do we expect the outputs of the model to be valid?" need to be
addressed. The performance of an AI/ML system will naturally degrade over time whether due to data drift, changes in

tists, engineers and operators go through the process of intent
capture, starting by revisiting the goals for the next flyby(s)
based on the downlink data and analysis.
Scientists and
engineers then have the opportunity to make changes to the
goals/plan while getting instant feedback on the viability of
their changes and on their impact on overall mission progress
and performance. The viability and impact analysis here is
based on an initial, low fidelity evaluation of possible onboard

the aggregated summary of all simulation runs for a given
task network
Mission impact tool—The mission impact UI view (shown in
Figure 10) provides an overview of the simulations spanning
the whole mission (that is, looking into all flybys) highlight-
ing, the impact of newly-added goals to the the progress and
success of the campaigns and to performance trends. This
view also shows how the plans perform with respect to key
performance indicators, and the uncertainty associated with

(e.g., time series)
• Resize/reshape/window
Feature Engineering
• Custom feature generation
• Auto feature generation
Data Visualization
• Spreadsheet viewer
• Basic plotting tools & viewer
• Scatter plot/time series
• Histogram (1D & 2D)
• Image viewer
Data Exploration
• Automatic data profiling
• Data type, range, stats
• Cardinality
• Correlation matrix
• Feature importance
• Univariate analysis
• Clustering
• Bias identification
Data Labeling
• Ontology definition
• Hand labeling
• Smart labeling

\section{Explanations on the management of ethical issues and data protection}

potentially threating our civilisation and even human survival [343]. A 
report on robotics and AI published by the British House of Commons 
highlighted specific ethical and legal issues, including transparent 
decision-making, minimising bias, accountability and privacy [325]. 
The first draft of the “Ethics Guidelines for Trustworthy AI” was pub­
lished by the European Commission’s High-Level Expert Group on 
Artificial Intelligence (‘AI HLEG’) [344]. According to the guidelines,

Progress in Aerospace Sciences 144 (2024) 100960
30
1. Ethical purpose: AI development, deployment and use should respect 
fundamental rights and applicable regulations as well as core prin­
ciples and values to ensure “ethical purpose";  
2. Technical robustness: AI should be technically robust and reliable 
since its use can cause unintentional harm, even in the presence of 
good intentions [344]. 
AI systems use large amounts of data, causing increasing concerns as

the smaller file sizes required for programming, which is becoming 
compatible with the uplink bandwidth of small satellites [264]. 
8.2. Ethical and legal challenges 
The use of AI in space systems raises a number of ethical and legal 
questions. Some researchers have identified the need for AI ethicists to 
help navigate where advances in this technology could lead [328]. This 
is more exciting because AI and space technologies offer a wide range of

text=Prof%20Stephen%20Hawking%2C%20one%20of,end%20of%20the%20h 
uman%20race.%22, 2014. 
[344] E. Commission, Have your say: European expert group seeks feedback on draft 
ethics guidelines for trustworthy artificial intelligence, Available at: https://ec. 
europa.eu/digital-single-market/en/news/have-your-say-european-expert-gro 
up-seeks-feedback-draft-ethics-guidelines-trustworthy, 2018. 
[345] S.T. Todd, J. Burke, Emerging legal issues in an AI-driven world, Available at:

[326] D. Harkut, K. Kasat, Introductory Chapter: Artificial Intelligence - Challenges and 
Applications, 2019. 
[327] J.B. James Manyika, The Promise and Challenge of the Age of Artificial 
Intelligence, 2018. 
[328] A. Pavaloiu, U. K¨ose, Ethical artificial intelligence - an open question, Journal of 
Multidisciplinary Developments 2 (04/01 2017) 15–27. 
[329] M.U. Scherer, Regulating artificial intelligence systems: risks, challenges, 
competencies, and strategies, Harv. J. Law Technol. 29 (2015) 353.

good intentions [344]. 
AI systems use large amounts of data, causing increasing concerns as 
more data is collected and used. Such high volumes and the level of 
dependence on such data will keep privacy at the forefront as one of the 
most significant legal issues to be addressed in the future. For instance, 
setting ethical parameters within which AI systems operate is para­
mount in tackling bias, considering the application of AI to data 
generated in space and prospective on-board AI space sector

potentially threating our civilisation and even human survival [343]. A 
report on robotics and AI published by the British House of Commons 
highlighted specific ethical and legal issues, including transparent 
decision-making, minimising bias, accountability and privacy [325]. 
The first draft of the “Ethics Guidelines for Trustworthy AI” was pub­
lished by the European Commission’s High-Level Expert Group on 
Artificial Intelligence (‘AI HLEG’) [344]. According to the guidelines,

There are times when robots are used to collect and process sensitive information, starting from personal data used in healthcare to 
proprietary information used in industrial settings. All such information should be protected from unauthorized access and cyber attacks 
[15]. Ensure safe communication channels between the robots and the cloud in order to protect against breaches of data or tampering.

Data Security
• Access management
• Sensitive info labeling
• User/group access rules
Data Version Control
• Original source
• Commit history with 
comments
Data Standardization
• Labeling standards
• Column naming standards
• Data type and unit standards
Data Security
• Access management
• Sensitive info labeling
• User/group access rules
Data 
Source
DB
DB
Query
Object
Graph
Fig. 1
Functional diagram of the Prepare step of the AI/ML workflow.

supporting the safety of orbiting spacecraft and debris mitigation [7].
With an exponentially growing number of space-related practical ser-
vices and research interests, a new focus has been appropriately made
on the defense and protection of spacecraft to ensure the continued
flow of information (Cukurtepe and Akgun [13], Jah [20], Brown, Cot-
ton, et al. [7], Contant-Jorgenson, Lála, Schrogl, et al. [11]). A few

Data Security
• Access management
• Sensitive info labeling
• User/group access rules
Data Version Control
• Original source
• Commit history with 
comments
Data Standardization
• Labeling standards
• Column naming standards
• Data type and unit standards
Data Security
• Access management
• Sensitive info labeling
• User/group access rules
Data 
Source
DB
DB
Query
Object
Graph
Fig. 1
Functional diagram of the Prepare step of the AI/ML workflow.

datasets via data governance. Central dataset repositories can be tied to enterprise authentication services such as Active
Directory to provide an easy-to-use and secure way for groups to manage access and modification permissions for their
datasets. For unrestricted datasets, a centralized repositry makes it easier for developers to find relevant datasets and
offers a single source of truth for datasets. Standards can also be implemented at the enterprise level to ensure data is

digital 
capabilities 
including 
software defined networks, data analytics, machine 
intelligence, cloud edge computing, and modular plug-n-
play systems.  Digital applies not only to our weapon 
systems but to our business processes as well, and the 
Space Force will apply similar techniques to enable a 
Digital Headquarters. Full implementation of our digital 
strategy will involve investments in Digital Engineering 
data and analytics infrastructure to ensure all our data is

data and analytics infrastructure to ensure all our data is 
discoverable, accessible, understandable, linked, and 
trusted across multiple security levels.  
Automation 
and 
autonomy 
will 
accelerate 
and 
streamline our operations and provide analytics to 
optimize mission and headquarters effectiveness. 
Applying machine learning and trusted levels of 
autonomy will allow our personnel to focus on data-
driven decision-making instead of manually sorting and

standards. 
4 
Many living beings could die or 
could be restricted for life; the 
environment could be damaged 
permanently. Loss of 
information which endangers 
the existence of the 
organization. Long-term 
unavailability of critical data or 
services without which the 
organization cannot function. 
The ML application is 
developed and documented 
with great care. Safety & 
Security is ensured with 
processes and techniques that 
go beyond traditional best 
practices and industry

intent, and updates that align evolving requirements.  We 
value clear verbal and written communication oriented to 
inform decisions and implement actions. 
We also value design approaches to critical thinking, and 
data-driven problem solving. Over the next year, the 
Director of Staff will publish guidance that establishes a 
standard 
for 
how 
space 
professionals 
approach 
structured data-driven decision-making. Like MTOs, a 
standardized process is not intended to constrain thinking

potentially threating our civilisation and even human survival [343]. A 
report on robotics and AI published by the British House of Commons 
highlighted specific ethical and legal issues, including transparent 
decision-making, minimising bias, accountability and privacy [325]. 
The first draft of the “Ethics Guidelines for Trustworthy AI” was pub­
lished by the European Commission’s High-Level Expert Group on 
Artificial Intelligence (‘AI HLEG’) [344]. According to the guidelines,

disaggregated and desynchronized bureaucratic process 
that increases risk for our Joint warfighters. We will 
consolidate and coordinate disparate processes to 
accelerate decisions and reduce that risk. 
 
Reducing bureaucracy does not mean eliminating the 
oversight required by law and policy. Rather, it emphasizes 
empowerment through delegation of decision authority to 
the most responsive competent authority, and a high 
degree of accountability. Tight alignment of responsibility,

post-processing, checking for violations such as data values
going over limits (e.g. power levels are too high), invalid
behaviors or combinations of behaviors (e.g. activity A can-
not ever overlap activity B), and for other types of undesired
situations. Further, ground systems often include automated
notifications to alert operators of an issue, given there may be
a very short turn around time for mission engineers to respond
to that issue and possibly prevent the loss of science data or

digitally supported decision-making throughout their 
careers.  
With digital engineering and fluency as foundational 
elements, we will drive Digital Operations across our 
space mission sets to increase all domain awareness and 
close the kill chain faster with more robust, informed C2 
decision options. In doing so, we will fully exploit modern 
commercially-based 
digital 
capabilities 
including 
software defined networks, data analytics, machine

potentially threating our civilisation and even human survival [343]. A 
report on robotics and AI published by the British House of Commons 
highlighted specific ethical and legal issues, including transparent 
decision-making, minimising bias, accountability and privacy [325]. 
The first draft of the “Ethics Guidelines for Trustworthy AI” was pub­
lished by the European Commission’s High-Level Expert Group on 
Artificial Intelligence (‘AI HLEG’) [344]. According to the guidelines,

the smaller file sizes required for programming, which is becoming 
compatible with the uplink bandwidth of small satellites [264]. 
8.2. Ethical and legal challenges 
The use of AI in space systems raises a number of ethical and legal 
questions. Some researchers have identified the need for AI ethicists to 
help navigate where advances in this technology could lead [328]. This 
is more exciting because AI and space technologies offer a wide range of

Despite the numerous advantages offered by AI in 
space exploration, there are challenges that need to 
be addressed. One significant concern is the 
reliability and robustness of AI systems operating in 
the harsh conditions of space. Ensuring the 
resilience of AI algorithms to radiation, extreme 
temperatures, and other space-specific challenges is 
crucial for the success of future missions. 
Additionally, ethical considerations surrounding AI 
decision-making in space, such as the potential for

[326] D. Harkut, K. Kasat, Introductory Chapter: Artificial Intelligence - Challenges and 
Applications, 2019. 
[327] J.B. James Manyika, The Promise and Challenge of the Age of Artificial 
Intelligence, 2018. 
[328] A. Pavaloiu, U. K¨ose, Ethical artificial intelligence - an open question, Journal of 
Multidisciplinary Developments 2 (04/01 2017) 15–27. 
[329] M.U. Scherer, Regulating artificial intelligence systems: risks, challenges, 
competencies, and strategies, Harv. J. Law Technol. 29 (2015) 353.

highlighting the open technological, ethical and legal challenges, as well 
as the ongoing efforts to overcome these challenges and to facilitate the 
uptake of AI technology in next-generation satellite systems. 
2. Spaceflight systems 
Thousands of active satellites are currently orbiting Earth and, in 
recent years [7], there has been an exponential growth of RSO [8], 
especially in the LEO environment [9]. Each satellite’s size, orbital 
parameters and configuration depend on its intended purpose. The

\section{Comment on resubmission (if applicable)}

changes into the main plan.
Then the team collectively
reviews these preliminary “low fidelity” outputs, implements
iterations as needed and approves advancing to the “high
fidelity” evaluation of possible outcomes and autonomy out-
put.
Such high fidelity evaluation is called here outcome
prediction phase.
At the outcome prediction phase, simulations produce a more
realistic and comprehensive view of the new plan’s impact
to the projected mission progress and performance. Itera-

be redirected to a previous step or workflow. Generally development begins with data preparation, and continues to
model selection, training, and evaluation. In the data preparation stage the model inputs should be kept in a feature store
that will also be accessible during the Deploy phase. A feature store is a centralized repository where you standardize
the definition, storage, and access of features for training and serving [34]. After evaluation the model should be

addition, new worlds, new science, and new phenomena to observe are appearing 
on the horizon. The new scientific goals and objectives often require multiple 
coordinating spacecraft to make simultaneous observations, or to detect events 
without ground intervention. This increase in the demands for new spacecraft has 
led to intense research and development efforts for the software applications and 
processes that are used during a space mission, both on ground, in the Mission

can be determined by performance of the model and/or a schedule.
After development, the artifacts are tested to ensure proper function and performance. Next a final review is conducted
before the packaged automated model pipeline is released for delivery to the target environment.
The major steps in the Develop phase are largely the same across the three environments and the differences at
workflow level have been discussed in the previous sections. The main differences are the types of artifacts built during

the ground (in this project we use the MEXEC planning and
execution system [4]) in nominal (or most likely) scenarios
to check for constraint violations. With respect to impact, the
small progress bars to the right of the campaign title shows
the impact of that goal in the overall mission compared to the
original set of goals.
It is important to note that these tools are domain depen-
dent, meaning that they are designed to support science goal
specification for a multi-flyby mission for a single spacecraft.

across those meetings, though we also encouraged them to
break into reflective discussions about the process and tools
along the way. We also collected feedback in the form of
survey responses in a journal they used for each day. We
had implemented the tools as click-through prototypes only,
and user study facilitator needed to “drive” the tools at the the
operator’s request. This dynamic evoked dialogue about what
information they needed to see and why.
14

eventually improve its decision-making capabilities. This type of feedback is 
identified as a reward, or reinforcement, and can be given either at the end of a 
series of actions or more frequently. Two main philosophies exist when considering 
reinforcement learning, passive and active reinforcement: in the first, the agent’s 
objective is to compute each states’ utility, while in the latter the agent must 
determine which actions to take. In general, anyway, the methodology used to

tists, engineers and operators go through the process of intent
capture, starting by revisiting the goals for the next flyby(s)
based on the downlink data and analysis.
Scientists and
engineers then have the opportunity to make changes to the
goals/plan while getting instant feedback on the viability of
their changes and on their impact on overall mission progress
and performance. The viability and impact analysis here is
based on an initial, low fidelity evaluation of possible onboard

benchmark cases in a dedicated analysis.
Benchmark.
This paragraph reports and sums up the tests carried out to
assess the performance of the models discussed up to now, against some simple
benchmarks, to confirm the effectiveness of the learning step and the reward
function design. Even if it may seem trivial, the first two comparisons are
against no-learning models, meaning that they have not gone through the
training procedure: the first simply propagates the free-dynamics starting from

the impact gap in green). Furthermore, the operators are also
presented with a few recommendations on how improve the
plan to avoid conflicts and aid in campaign success. These
recommendations are essential in fitting in with the iterative
workflow of plan development.
Downlink analysis
Subsystem Downlink Analysis Tool—The subsystem downlink
analysis tool (shown in Figure 11) allows operators to review
Figure 10. Mission Impact tool: shows an overview of the

changes into the main plan.
Then the team collectively
reviews these preliminary “low fidelity” outputs, implements
iterations as needed and approves advancing to the “high
fidelity” evaluation of possible outcomes and autonomy out-
put.
Such high fidelity evaluation is called here outcome
prediction phase.
At the outcome prediction phase, simulations produce a more
realistic and comprehensive view of the new plan’s impact
to the projected mission progress and performance. Itera-

of the outcome of the plan, while higher fidelity predictions,
which we made available later in the activity, provided es-
timations based on a Monte Carlo Simulation approach that
modeled the outcomes across 10,000 scenarios/outcomes.
While participants generally accepted the progression from
low fidelity to high fidelity data, conversations with scientist
participants revealed that some wanted higher fidelity data
available at the beginning of negotiations. In this particular

that the 35% of the new generation is formed by individuals picked from the old 
generation. The selected value ensures a balance between effectiveness of the 
search (lower elite population fractions) and survival of fit individuals (higher elite 
population fractions). 
8.5 Results 
The investigation on methodologies to improve and automate the space mission and 
spacecraft design is a vast effort, branching out into many fields of science and

The output of the factor graph is a maximum-likelihood
estimate of the state variables considered, and the marginal
distribution of each variable. In future work, this data will be
displayed in the Subsystem Downlink Analysis Tool (Figure
11), providing operators with key insight into unmeasured
variables and, critically, with the likelihood of each consid-
ered hypothesis - helping operators assess the state of the
spacecraft and understand why autonomy made its decisions.
6. USER STUDY
Study Design

Date: 07/23 
Revision: v4 
 
 
 
As Published in “Precision Medicine for Long and Safe Permanence of Humans in Space” 
7 
Current AI Technology in Space 
 
 
 
Figure 2: Comparison of Computational Density Per Watt of State-of-the-art Rad-Hard Processors (BAE RAD750, 
CAES Gaisler GR740, and BAE RAD5545) and Commercial Embedded Processors (Xilinx Zynq 7020 and Intel 
Core i7-4610Y) [21] 
Likewise, the power efficiency of rad-hard processors, which can be estimated from the

be redirected to a previous step or workflow. Generally development begins with data preparation, and continues to
model selection, training, and evaluation. In the data preparation stage the model inputs should be kept in a feature store
that will also be accessible during the Deploy phase. A feature store is a centralized repository where you standardize
the definition, storage, and access of features for training and serving [34]. After evaluation the model should be

aspects, but, among all, is mainly described by the network architecture and
the state-action spaces; on the other side, the reward formulation embeds the
mission objectives. Once these two, or more if needed, cases have been selected,
the workflow proposes to train and test them with different levels of initial
conditions. This difference can be interpreted in distinct ways: for instance, in
Section 4.3, it concerns the randomness level of the initial conditions; while, in

learning algorithms and problems: there is a component of an algorithm to be 
improved; the agent possesses prior knowledge; data is represented in a specific 
way; a feedback action provides guidance during learning. When a specific 
algorithm needs to learn from its surrounding world, three main learning algorithms 
are available to the designer, and will be discussed later: reinforcement learning, 
supervised learning, unsupervised learning. 
Learning from examples

Adapt-
ability
The pre-trained model may not
adapt optimally to the specifics of
the target task.
Data Efficiency
It often requires fewer samples to
adapt to the target task due to prior
knowledge.
Risk of Negative
Transfer
In some cases, transferring knowl-
edge may harm the performance of
the target task if the source task is
too dissimilar.
Table 4.11: Advantages and disadvantages of transfer learning and fine-tuning techniques for deep reinforcement learning.
99

characteristics of this method, that distinguish it from all the others, are the
trial-and-error search and the reward. In fact, the learning agent starts without
any knowledge about the optimal action to take for a particular problem, and
therefore it is forced to discover which are the ones that yield the greatest reward
number. Each action taken may affect not only the immediate reward but
also the future development of the problem and, then, the overall sequence of

have to negotiate a future that is getting more and more crowded with 
familiar, unfamiliar and unfriendly players all vying for technological 
and strategic superiority. Threats to the space realm and its sustaining 
infrastructure have escalated as a result of this change [318]. Ground 
and aerospace system architectures must offer a high level of resilience 
in order to ensure mission success. Resilience is therefore a crucial 
design factor that should be traded off against cost and capabilities when

disaggregated and desynchronized bureaucratic process 
that increases risk for our Joint warfighters. We will 
consolidate and coordinate disparate processes to 
accelerate decisions and reduce that risk. 
 
Reducing bureaucracy does not mean eliminating the 
oversight required by law and policy. Rather, it emphasizes 
empowerment through delegation of decision authority to 
the most responsive competent authority, and a high 
degree of accountability. Tight alignment of responsibility,

resourcing and oversight functions. 
 
To ensure our force design offers the Joint Force 
assured effects, the SWAC will analyze opportunities 
to enhance the resilience of legacy systems as an 
interim step to fielding a force designed to operate in 
a warfighting domain. The SWAC will develop future 
force 
structures 
that 
meet 
evolving 
mission 
requirements, are resilient to the threat, and are cost-
informed. The SWAC will execute Service wargaming

changes into the main plan.
Then the team collectively
reviews these preliminary “low fidelity” outputs, implements
iterations as needed and approves advancing to the “high
fidelity” evaluation of possible outcomes and autonomy out-
put.
Such high fidelity evaluation is called here outcome
prediction phase.
At the outcome prediction phase, simulations produce a more
realistic and comprehensive view of the new plan’s impact
to the projected mission progress and performance. Itera-

array of high-fidelity simulations. The collected predicted
outcomes can then be used by the uplink team to not only
observe the expected execution, but also attach confidence
values to the various goals and activities within the generated
plans. As such, repeated simulation runs and collection of the
outcomes fit within the proposed iterative workflow of uplink
operations, which ultimately serves the goal of increasing
the confidence of the uplink team in the expected behavior

\section{Bibliography (max. 15 references, not included in character limits)}

of the main research topic, corroborating its procedures and techniques. All
three cases confirmed the good quality of the proposed methodology, achieving
satisfying performance levels in each of the applications’ frameworks.
Working for a Ph.D. is not an easy road, it is a path full of stops, sudden turns,
backs out and comebacks, and surely does not exhaustively explore a research
field. But certainly, it helps pushing towards the accomplishment of evermore

supporting the safety of orbiting spacecraft and debris mitigation [7].
With an exponentially growing number of space-related practical ser-
vices and research interests, a new focus has been appropriately made
on the defense and protection of spacecraft to ensure the continued
flow of information (Cukurtepe and Akgun [13], Jah [20], Brown, Cot-
ton, et al. [7], Contant-Jorgenson, Lála, Schrogl, et al. [11]). A few

challenges and unlocking new opportunities. By 
leveraging AI techniques, scientists and engineers 
can enhance data analysis, enable autonomous 
systems, improve navigation, and achieve greater 
efficiency in space missions. 
 
Methodology: 
To conduct this research initially, an extensive 
literature search was performed using reputable 
scientific databases, including IEEE Xplore, ACM 
Digital Library, and Google Scholar. The search 
keywords included "artificial intelligence," "AI,"

to the reader. We decided to give importance to work published in the last few years,
avoiding the historical perspective of older and well-established fundamental works.
Additionally, we decided to avoid publications which are strongly speculative in
nature: while visionary ideas are interesting to follow, an important requirement for
this survey was a well-motivated applicability for a space-related challenge, ideally
inspired by an already established or newly proposed mission concept. This narrow

address 
the 
challenges 
and 
opportunities in integrating AI in the field of space 
exploration. 
Furthermore, the selected papers were scrutinized to 
extract relevant data and insights that could 
contribute to the research paper. This involved 
identifying statistical information, experimental 
results, case studies, and any other data that 
supported the findings and conclusions of the 
respective studies. The extracted data and insights 
were then used to strengthen the research paper's

of the main research topic, corroborating its procedures and techniques. All
three cases confirmed the good quality of the proposed methodology, achieving
satisfying performance levels in each of the applications’ frameworks.
Working for a Ph.D. is not an easy road, it is a path full of stops, sudden turns,
backs out and comebacks, and surely does not exhaustively explore a research
field. But certainly, it helps pushing towards the accomplishment of evermore

Source
DB
DB
Query
Object
Graph
Fig. 1
Functional diagram of the Prepare step of the AI/ML workflow.
framework. A number of FOSS and COTS solutions exist and are implemented either as tools or APIs at the project
level (e.g., DVC∗or the tf.data.Dataset module†) or as an enterprise-level central dataset repository (e.g., Collibra‡).
Beyond storing the data, centralized dataset repositories provide additional features that aid in the management of

Eirates), Northrop Grumman Corporation (United States), and the 
SmartSat Cooperative Research Centre (Australia) for their support of 
this work through the Grant No. FSU-2022-013, the Collaborative 
Research Project No. RE-04143, and the Doctoral Research Project No. 
2.13s, respectively. The authors would also like to thank Dr Andoh Afful, 
Dr Suraj Bijjahalli and Prof. Wei Xiang and Mr Thomas Fahey for their 
insightful feedback, which helped to improve the quality of this article. 
References

Analysis
Once we concluded the study, we transcribed the audio from
the sessions and compiled the survey responses. To identify
themes, we grouped clusters of related observations into
affinity groups, focusing on topics that related to our research
questions.
Findings
In this section, we describe preliminary themes that emerged
from our analysis.
The user study participants in general
successfully used the tools and series of steps to complete
their operations tasks. The operators’ participation, inquiries,

the mission’s objectives.
• Initial conditions robustness: where all the parameters involved in
the random selection and their relative randomness level are defined or
designed.
The adopted workflow based on these three aspects is schematized in Fig. 6.1.
This methodology is proposed as a sort of vademecum or guidelines to follow
for the implementation and testing of a robust DRL-based path or strategy
planning agent. Indeed, all three scenarios that will be analysed in the next

were then used to strengthen the research paper's 
arguments and provide evidence for the benefits and 
applications of AI in the field of space. 
Overall, the literature search and analysis process 
involved 
thorough 
exploration 
of 
reputable 
scientific databases, careful selection of articles 
based 
on 
specific 
keywords, 
review 
of 
methodologies employed in the selected studies, and 
extraction of relevant data and insights. This

Analysis
Once we concluded the study, we transcribed the audio from
the sessions and compiled the survey responses. To identify
themes, we grouped clusters of related observations into
affinity groups, focusing on topics that related to our research
questions.
Findings
In this section, we describe preliminary themes that emerged
from our analysis.
The user study participants in general
successfully used the tools and series of steps to complete
their operations tasks. The operators’ participation, inquiries,

of the main research topic, corroborating its procedures and techniques. All
three cases confirmed the good quality of the proposed methodology, achieving
satisfying performance levels in each of the applications’ frameworks.
Working for a Ph.D. is not an easy road, it is a path full of stops, sudden turns,
backs out and comebacks, and surely does not exhaustively explore a research
field. But certainly, it helps pushing towards the accomplishment of evermore

challenges and unlocking new opportunities. By 
leveraging AI techniques, scientists and engineers 
can enhance data analysis, enable autonomous 
systems, improve navigation, and achieve greater 
efficiency in space missions. 
 
Methodology: 
To conduct this research initially, an extensive 
literature search was performed using reputable 
scientific databases, including IEEE Xplore, ACM 
Digital Library, and Google Scholar. The search 
keywords included "artificial intelligence," "AI,"

to the reader. We decided to give importance to work published in the last few years,
avoiding the historical perspective of older and well-established fundamental works.
Additionally, we decided to avoid publications which are strongly speculative in
nature: while visionary ideas are interesting to follow, an important requirement for
this survey was a well-motivated applicability for a space-related challenge, ideally
inspired by an already established or newly proposed mission concept. This narrow

the output of an ANN to a reference model. The algorithm iteratively compares the 
output of the network to the model, and by applying a corrective action on the 
network weights and biases, the output is adapted to match the desired one. The 
training is generally based on previous experience, although methods that modify 
the parameters of the network exist. Three types of learning algorithms have been 
developed. 
Supervised learning denotes a method in which some input vectors (training

standing, should be investigated first. Classical and specialised methods, 
on the other hand, are often naive, whereas heuristic and metaheuristic 
paradigms can be utilised to various conditions. One key advantage of 
heuristic and metaheuristic paradigms is their robustness. In this 
context, robustness refers to an algorithm’s ability to solve a wide range 
of problems and even multiple sorts of problems, with only slight 
changes to account for each problem’s specific properties. A stochastic

4.6.
Closing Remarks
extensively tested to assess the model robustness and sensitivity, aimed at
specifically analysing the case of uncertainty in the state and how it affects the
performance of the algorithm. The first results showed how the main models are
not robust to the noisy input state. Therefore, to overcome this problem, two
different methods have been proposed and analysed: a re-training procedure
and a transfer-learning procedure. Both of them improve the capabilities of

go in and outputs come out and there is little insight into how the model is making its determination. The seemingly
non-deterministic nature of AI/ML systems may cause some end users to be hesitant to trust AI/ML systems. AI/ML
systems will also need to meet more standard software metrics for reliability and security as they suffer many of the
same vulnerabilities that standard software systems do. There needs to be procedures in place to mitigate radiation

prone to bias and can be exploited to generate damaging or misleading 
resources. It is critical to be aware of these hazards and to take pre­
cautions to mitigate them. ANN is a type of AI that tries to replicate the 
way the human brain works. The processing units are ANN that are 
composed of inputs and outputs. ANN are a kind of ML technology that is 
inspired by biology and is supposed to work in way similar to the brain 
(loosely). Fig. 21 depicts the main types of NN types [3,81].

of the main research topic, corroborating its procedures and techniques. All
three cases confirmed the good quality of the proposed methodology, achieving
satisfying performance levels in each of the applications’ frameworks.
Working for a Ph.D. is not an easy road, it is a path full of stops, sudden turns,
backs out and comebacks, and surely does not exhaustively explore a research
field. But certainly, it helps pushing towards the accomplishment of evermore

defined as Target (TRG). In this thesis, the chaser represents the inspecting
spacecraft and the target is the unknown and uncooperative object. During
the work, the following four reference frames will be used as defined in [27]:
• ECI-reference frame. This is the system which has the origin in the
centre of the Earth and the plane as the Earth’s equator (Earth-centerd
Inertial (ECI)). The ˆI is directed along the vernal equinox, and the ˆK

Eirates), Northrop Grumman Corporation (United States), and the 
SmartSat Cooperative Research Centre (Australia) for their support of 
this work through the Grant No. FSU-2022-013, the Collaborative 
Research Project No. RE-04143, and the Doctoral Research Project No. 
2.13s, respectively. The authors would also like to thank Dr Andoh Afful, 
Dr Suraj Bijjahalli and Prof. Wei Xiang and Mr Thomas Fahey for their 
insightful feedback, which helped to improve the quality of this article. 
References

Chapter 2.
Background & State-of-the-Art
ˆi = r
r, ˆk = h
h, ˆj = ˆk ׈i
where h is the angular momentum. In this way, it is possible to declare the
chaser-target relative position both in ECI:
δr = rTRG −rCHS
(2.4)
and in LVLH reference frames:
ρ = xˆi + yˆj + zˆk
(2.5)
Then, subtracting Eq. 2.3 to Eq. 2.2, and substituting Eq. 2.4, the following
relative acceleration can be retrieved in ECI reference frame:
δ¨r = −µ(rCHS + ρ)
∥rCHS + ρ∥3 + µrCHS
r3
CHS
(2.6)

Analysis
Once we concluded the study, we transcribed the audio from
the sessions and compiled the survey responses. To identify
themes, we grouped clusters of related observations into
affinity groups, focusing on topics that related to our research
questions.
Findings
In this section, we describe preliminary themes that emerged
from our analysis.
The user study participants in general
successfully used the tools and series of steps to complete
their operations tasks. The operators’ participation, inquiries,

\end{document}