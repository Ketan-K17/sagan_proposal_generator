\documentclass{article}
\usepackage{amsmath}
\usepackage{geometry}
\geometry{margin=1in}
\title{Advanced Rocket Engines Development: A Research Project}
\author{}
\date{}

\begin{document}

\maketitle

\section{Introduction: Originality of the Research Project}

The University of Luxembourg, through its Research Unit in Engineering Science (RUES), is committed to addressing the socio-economic needs and challenges of society and industry by becoming a leader in education and research within the Greater Region and globally. The unit focuses on integrating research and education to develop future leaders and critical thinkers. Its research activities are organized into three main areas: Construction and Design, Energy and Environment, and Automation and Mechatronics. The university collaborates with over 70 private and public organizations through SnT’s Partnership Programme, tackling key challenges in ICT and fostering a dynamic interdisciplinary research environment. The university emphasizes the importance of disseminating research outputs to industry, policymakers, and society, encouraging activities that generate impact from the initial project planning stage. This aligns with the European Strategic Technology Plan and the EU's Innovation Union goals, aiming to create an innovation-driven research environment.

The FNR places significant emphasis on the impact of research outputs on science, industry, policy making, and society. To maximize this impact, FNR-funded research results are expected to be disseminated through high-quality, Open Access publications, in line with the FNR Policy on Open Access. The FNR also supports the deposition of preprints in open access repositories and offers an Open Access Fund to cover publication costs. Additionally, the FNR encourages activities aimed at generating impact from the initial project planning stage, including the dissemination of research to the general public and media.

The SnT’s Partnership Programme exemplifies this approach by collaborating with over 70 private and public organizations to tackle key challenges in ICT. Since its inception in 2009, SnT has rapidly developed, launching over 100 EU and ESA projects, protecting and licensing IP, and creating a dynamic interdisciplinary research environment. This initiative aligns with the broader goal of becoming a leader in education and research in the Greater Region and globally, with a focus on energy, environment, and sustainable growth, contributing to the European Strategic Technology Plan and the Innovation Union.

LuxProvide complements these efforts by offering a platform that combines data science and supercomputing resources, aiding research and business players in Luxembourg and the Greater Region. Their approach emphasizes design thinking and co-creation, ensuring effective innovation. Public-private partnerships further enhance this ecosystem, allowing researchers to work closely with companies and public partners, spending a significant portion of their research period within the company to achieve innovation and optimization goals. This collaborative model ensures that research outputs are not only impactful but also directly applicable in real-world environments.

The Greater Region is positioning itself as a leader in education and research, with a global focus on core areas such as energy, environment, and sustainable growth. This aligns with the European Strategic Technology Plan and the EU's Innovation Union initiative. The aim is to create an innovation-driven research environment that integrates research and education, preparing future industry leaders, policymakers, and society at large. Applicants are encouraged to demonstrate the value and impact of their research outputs, including publications, data, and intellectual property, while also engaging with the public and media.

The SnT Partnership Programme exemplifies this approach by collaborating with over 70 private and public organizations to tackle ICT challenges. Since its inception in 2009, SnT has rapidly developed, launching numerous EU and ESA projects, protecting intellectual property, and fostering a dynamic interdisciplinary research environment.

LuxProvide complements these efforts by offering a platform that combines data science and supercomputing resources, supporting both research and business players in Luxembourg and the Greater Region. Their approach emphasizes design thinking and co-creation to drive effective innovation.

Creaction, as an ESA BASS broker, assists Luxembourg-based start-ups and SMEs in integrating space applications to meet their innovation needs. They offer a holistic suite of services for product and company development, from ideation to commercialization, and organize workshops to pre-incubate innovation projects.

Overall, these initiatives collectively contribute to a robust ecosystem that supports innovation, research, and development, fostering sustainable growth and technological advancement in the Greater Region and beyond.

The integration of accumulated knowledge and heritage information is crucial for the efficient preparation and early study phases of projects. Experts emphasize the need for quick access to reliable and synthesized information, as heritage knowledge aids in better parameter estimations and accelerates the study kick-start. Knowledge Graphs (KGs) are gaining momentum in both academic and industrial fields due to their ability to handle diverse data, which is essential for new team members to get up to speed quickly. Young professionals often face challenges in accessing data compared to their more experienced colleagues, highlighting the importance of efficient information retrieval systems.

Experts spend a significant amount of work time searching for information, with colleagues being the preferred source due to their ability to provide quick answers and tacit knowledge. However, human experts can have biases or forget information, making past mission reports and textual documents valuable sources of heritage information. Internal databases are less favored compared to online materials, partly due to maintainability issues.

In the realm of scientific research, the focus should be on the content and quality of outputs rather than the quantity or publication metrics. Diverse outputs, including data, software, and mentoring, are important across disciplines. Researchers are encouraged to list the value and impact of all research outputs, including preprints and intellectual property, and to engage in activities that generate impact from the project's inception.

The SnT Partnership Programme exemplifies collaboration between researchers and over 70 private and public organizations, addressing key challenges in ICT. Since its launch in 2009, the Centre has rapidly developed, recruiting top scientists, launching numerous projects, and creating a dynamic interdisciplinary research environment. This approach underscores the importance of disseminating research to the general public and media, ensuring that scientific advancements have a broad societal impact.

The FNR emphasizes the importance of research outputs impacting science, industry, policy makers, and society. Applicants are expected to demonstrate the value and impact of their research outputs, which include preprints, publications, data, reagents, software, intellectual property, and the training of young scientists. The FNR encourages the dissemination of research to the public and media, and activities aimed at generating impact should be planned from the project's inception. Research outputs should go beyond traditional articles to include data, software, mentoring, and societal outreach. The focus should be on the quality and content of scientific outputs rather than their quantity or publication venue. The FNR is sensitive to legitimate delays in research activity due to personal factors like parental leave or disability. Outputs vary across disciplines and individuals, and important contributions may extend beyond research articles. The FNR requires that research results be disseminated through high-quality, Open Access publications, with costs potentially covered by the FNR’s Open Access Fund.

\section{Hypothesis, Research Objectives, and Envisaged Methodology}

The FNR-funded research activities are expected to adhere to several key principles that emphasize the importance of ethical conduct, research integrity, and the impact of research outputs. Applicants must comply with the FNR Research Integrity Guidelines, ensuring that their work respects fundamental ethical principles, such as those outlined in the Charter of Fundamental Rights of the European Union. Any research misconduct, including non-compliance with ethical regulations, provision of false information, plagiarism, or data falsification, may lead to proposal rejection and further actions by the FNR.

The focus of research evaluation should be on the content and quality of scientific outputs rather than the number of publications, the venue of publication, or aggregate metrics. It is crucial to consider a diverse range of research-related and non-research-related outputs, which can vary across disciplines and individuals. These outputs may include data, reagents, software, mentoring, societal outreach, intellectual property, and policy changes. The FNR encourages the dissemination of research to the general public and media, highlighting the importance of generating impact from the initial project planning stage.

Furthermore, the FNR acknowledges legitimate delays in research activities due to personal factors such as parental leave, part-time work, or disability, which may affect an applicant's research output record. Overall, the FNR places great importance on the value and impact of research outputs on science, industry, policymakers, and society at large.

The FNR, as a signatory of the DORA Declaration, emphasizes the importance of evaluating research proposals based on the quality and impact of scientific outputs rather than relying on journal-based metrics like Journal Impact Factors. Applicants are encouraged to list a diverse range of research outputs, including datasets, software, intellectual property, and the training of young scientists. The focus should be on the scientific content and quality of these outputs, rather than their quantity or the prestige of the publication venue. This approach aligns with fostering a positive research culture that values diversity and inclusion, encouraging supervisors and mentors to develop gender equity plans and set relevant KPIs. The FNR also supports the dissemination of research to the general public and media, highlighting the need for activities aimed at generating impact from the initial project planning stages. Reviewers are instructed to evaluate proposals based solely on the information submitted by the applicant, including the proposal, attachments, and ORCID profile, while considering the diverse range of research-related and non-research-related outputs. This comprehensive evaluation process ensures that all types of research contributions are valued, promoting a more inclusive and effective research environment.

Our institution is dedicated to seamlessly integrating research and education to cultivate future leaders and critical thinkers. Our research activities are organized into three main areas: Construction and Design, focusing on civil and mechanical engineering structures; Energy and Environment, emphasizing energy efficiency and renewable energies; and Automation and Mechatronics. We collaborate with over 70 private and public organizations through SnT’s Partnership Programme, addressing key challenges in ICT and launching numerous projects, including EU and ESA initiatives. Our research outputs, which include data, reagents, software, and training, are expected to demonstrate value and impact across industry, policy makers, and society. The FNR encourages the dissemination of research to the public and media, emphasizing the importance of content and quality over publication metrics. Our research is supported by major space projects like MILAN, KM4SR, ECOSTRESS, CRISTAL, and PUBLIMAPE, which explore areas such as machine learning for astronomy and ecosystem monitoring. We prioritize a dynamic interdisciplinary environment, fostering innovation and collaboration to address global challenges.

Applicants seeking research grants are expected to demonstrate the value and impact of their research outputs, including preprints, publications, data, and software, while also engaging in public dissemination activities. The FNR emphasizes the importance of planning for impact from the project's inception and encourages the use of open access repositories for preprints, supported by the FNR’s Open Access Fund. A Data Management Plan (DMP) is mandatory for all projects, ensuring data is stored, curated, and made accessible in line with the FNR Policy on Research Data Management. Applicants, with their host institutions, must establish and regularly update the DMP, deposit data in trusted archives, and provide free access to it. The FNR uses applicant data in compliance with Luxembourg's data protection laws, including anonymized data for statistical purposes. Additionally, consultancy services are available to support hi-tech and telecommunication projects through all stages, from initial studies to project completion. The integration of AI in space mission design and the use of metadata for heritage analysis are also highlighted as critical components in research and project planning.

The feasibility study process, as illustrated by the Spiral Model in Figure 2.6, is an iterative design approach that involves several key stages: preparation, the study itself, and post-study activities. This process is integral to the Concurrent Engineering approach, often employed at the European Space Agency's Concurrent Design Facility (ESA CDF). The study phase typically involves the full design team and may require multiple sessions, each lasting half a day, to explore various design options and iterations. The feasibility study outputs include a study report, an Engineering Model (EM), and sometimes a record of lessons learned. These outputs are crucial for capturing unstructured data, such as preliminary requirements, conceptual designs, and operational concepts, which are not stored according to a structured data model.

The study report, a primary output of the feasibility study, summarizes mission objectives, requirements, and design drivers, with the payload often being the main design driver. Initial requirements may be modified based on new findings during the study phase, which also aims to determine the necessary number of experts and design sessions to achieve the expected results within a limited timeframe. The Engineering Models, as discussed in Chapter 6, can be reused and integrated into a database to support current design efforts, as suggested by expert survey findings.

The thesis concludes by evaluating the findings and contributions, addressing the aims and objectives, and discussing the limitations of the methods applied. Recommendations for future work include developing a Design Engineering Assistant (DEA) architecture, leveraging Natural Language Processing (NLP) and other advanced methodologies to enhance the conceptual design phase of complex engineering systems.

\section{Expected Outcomes / Impact}

The FNR places significant emphasis on the impact of research outputs on science, industry, policy making, and society at large. To maximize this impact, FNR-funded research results are expected to be disseminated through high-quality, Open Access publications, in line with the FNR Policy on Open Access. The FNR also supports the deposition of preprints in open access repositories and encourages the dissemination of research to the general public and media. Applicants are expected to list the value and impact of all research outputs, including preprints, publications, data, reagents, software, intellectual property, and the training of young scientists. The FNR discourages the use of journal-based metrics, such as Journal Impact Factors, as a measure of quality, urging a focus on scientific content instead. Public-private partnerships are encouraged, with a significant portion of research time spent in industry settings. The FNR is committed to the principles of the European Charter for Researchers and requires applicants and host institutions to agree to its terms and conditions. The evaluation of proposals involves a panel of generalists who review and recommend projects for funding, ensuring alignment with the FNR's goals of impactful research dissemination and societal benefit.

The Fonds National de la Recherche (FNR) places significant emphasis on the impact of research outputs on science, industry, policy making, and society at large. To maximize this impact, FNR-funded research results are expected to be disseminated through high-quality, Open Access publications, in line with the FNR Policy on Open Access. The FNR supports the dissemination of research to the general public and media, and encourages the deposition of preprints in open access repositories. Additionally, the FNR promotes the protection and economic exploitation of research results, expecting host institutions to have appropriate intellectual property protection and exploitation strategies in place. Applicants are required to list the value and impact of all research outputs, including preprints, publications, data, reagents, software, and the training of young scientists. The focus is on the content and quality of scientific outputs rather than their quantity or the venue of publication. The FNR also supports the diverse range of research-related and non-research-related outputs, recognizing that important contributions may extend beyond traditional research articles. Through initiatives like SnT’s Partnership Programme, researchers collaborate with numerous private and public organizations to address key challenges in ICT, fostering a dynamic interdisciplinary research environment.

The FNR emphasizes the importance of focusing on the content and quality of scientific outputs rather than their quantity, publication venue, or journal metrics. It encourages a comprehensive evaluation of research-related and non-research-related outputs, recognizing that important contributions vary across disciplines and individuals. These outputs can include data, reagents, software, mentoring, leadership, societal outreach, intellectual property, and policy changes. Researchers are expected to list the value and impact of all outputs, including preprints and publications, and to plan for dissemination to the public and media from the project's inception. The FNR values the impact of research on science, industry, policy, and society, requiring results to be published in high-quality, Open Access formats, with costs potentially covered by the FNR’s Open Access Fund. Collaboration with over 70 organizations through SnT’s Partnership Programme highlights the dynamic research environment, fostering interdisciplinary work and addressing industry and public sector challenges. FNR-funded research must adhere to ethical principles and integrity guidelines, with a focus on diversity, inclusion, and gender equity. Financial support from FNR should be acknowledged in all communications, and research activities should ensure transparency and favorable working conditions.

The FNR, as a signatory of the DORA Declaration, emphasizes the evaluation of research proposals based on the quality and impact of scientific outputs rather than relying on journal-based metrics like Journal Impact Factors. This approach encourages applicants to highlight a diverse range of research-related and non-research-related outputs, including preprints, research publications, data, reagents, software, intellectual property, and the training of young scientists. The FNR also values the dissemination of research to the general public and media, urging applicants to plan for impact-generating activities from the project's inception. Reviewers are instructed to consider only the information submitted by the applicant, including the proposal, attachments, and ORCID profile, ensuring a comprehensive assessment of the applicant's contributions beyond traditional research articles.

The FNR emphasizes the importance of a comprehensive approach to evaluating research outputs, extending beyond traditional research articles to include data, reagents, software, mentoring, group leadership, societal outreach, intellectual property, and policy changes. It is crucial to focus on the content and quality of scientific outputs rather than their quantity or the prestige of the publication venue. Researchers are encouraged to consider the diverse range of outputs that vary across disciplines and individuals. The FNR also recognizes the impact of personal factors such as parental leave, part-time work, and disabilities on research productivity and output.

To maximize the impact of research on science, industry, policy-making, and society, the FNR requires that results from funded research be disseminated through high-quality, Open Access publications, in line with its Open Access Policy. The FNR supports this by offering refunds for project-related publication costs through its Open Access Fund. Additionally, the FNR encourages the dissemination of research to the general public and media, highlighting the need for impact-generating activities to be integrated into project planning from the outset.

Compliance with the FNR Research Integrity Guidelines is mandatory, and any research misconduct, such as non-compliance with ethical regulations, provision of false information, plagiarism, or data falsification, may lead to proposal rejection and further actions. The FNR is committed to fostering a research culture that values integrity and the broad impact of research outputs.

\section{Explanations on the Management of Ethical Issues and Data Protection}

The FNR-funded research activities, both inside and outside academia, must adhere to several general principles to ensure ethical and impactful research. All research activities should respect fundamental ethical principles, as outlined in the Charter of Fundamental Rights of the European Union, and comply with the FNR Research Integrity Guidelines. Any research misconduct, such as non-compliance with ethical regulations, provision of false information, plagiarism, or data falsification, may lead to proposal rejection and further actions by the FNR.

Host institutions are responsible for obtaining all necessary authorizations from ethical and data protection committees or other regulatory bodies. Any ethical misconduct can result in the immediate suspension or termination of the grant, with potential additional sanctions, including reimbursement requests and legal actions.

The FNR emphasizes the importance of research impact on science, industry, policy makers, and society. Applicants are expected to demonstrate the value and impact of their research outputs, including preprints, publications, data, reagents, software, intellectual property, and the training of young scientists. Dissemination of research to the public and media should be planned from the project's inception.

Beneficiaries must comply with the FNR research integrity guidelines and ethical charter, available on the FNR website. The FNR endorses the European Code of Conduct for Research Integrity and the Singapore Statement on Research Integrity. The merit review process follows international standards of transparency, impartiality, confidentiality, and integrity, as defined by the Global Summit on Merit Review.

Panel experts are required to read the FNR Ethics Charter and Code of Conduct for Research Assessment, sign a Participation Form, and thoroughly review assigned proposals. They should consider factors affecting research activity, such as personal leave or part-time work, and ensure all electronic materials acknowledge FNR funding with the appropriate logo and project code.

The FNR has established a comprehensive framework to ensure the confidentiality, ethical compliance, and effective evaluation of research proposals. Panel members and experts involved in the evaluation process must declare their commitment to confidentiality and are prohibited from using the data for personal purposes. They are required to read relevant documents, sign a participation form, and thoroughly review assigned proposals. The applicant and Host Institution (HI) agree to the inclusion of certain information in FNR publications and acknowledge that project abstracts, devoid of confidential information, may be shared with external experts. In case of conflicts of interest, panel members must withdraw from their tasks, and any inability to fulfill obligations must be reported to the FNR immediately. The FNR emphasizes the importance of ethical compliance, requiring all necessary authorizations and addressing any misconduct with potential sanctions. Beneficiaries have rights under GDPR to access and modify their personal data. The FNR is committed to maximizing the impact of research outputs and may implement measures to ensure proper documentation and development plans. In case of project delays or inability to continue, beneficiaries must inform the FNR promptly, providing full justification and an estimated timeline for resumption.

The successful execution of research projects requires access to essential resources such as laboratories and libraries. The applicant and the Host Institution (HI) must agree to the electronic storage and use of data by the FNR, adhering to Luxembourg's data protection laws. This includes the use of anonymized data for statistical purposes. Good practices in data management, protection, and security are crucial, and researchers must follow the FNR Policy on Research Data Management, complying with the FAIR principles. Under the EU’s General Data Protection Regulation (GDPR), beneficiaries have the right to access and modify their personal data, with requests to be made in advance to the FNR.

The FNR emphasizes the importance of research outputs impacting science, industry, policy, and society. To maximize impact, a data management plan must be established and regularly updated. Data should be deposited in a trusted archive and made accessible according to the principle of "as open as possible, as closed as necessary," unless it conflicts with legitimate interests or obligations. Applicants must permit the FNR to share application data for evaluation and management purposes.

All FNR-funded projects must comply with GDPR, ensuring necessary authorizations are obtained. Confidentiality is paramount, with panel members and experts required to treat data confidentially. The FNR may publish certain information, such as the beneficiary's name and nationality, post-evaluation. Innovative solutions are encouraged to protect data integrity, especially in distributed environments, supporting informed decision-making in areas like climate change and security. Data sharing is often hindered by confidentiality and data silos, but projects like the DEA aim to demonstrate the benefits of knowledge reuse.

The applicant, supported by their host institution, is required to ensure the successful execution of their research project by utilizing necessary resources such as laboratories and libraries. Both the applicant and the host institution must agree to the electronic storage and use of data by the FNR, adhering to Luxembourg's data protection laws. This includes the use of anonymized data for statistical purposes. Researchers must follow good practices in data management, protection, and security, aligning with the FNR Policy on Research Data Management and the FAIR principles. A data management plan should be established and regularly updated, ensuring data is deposited in a trusted archive and made accessible according to the principle of "as open as possible, as closed as necessary," unless it conflicts with legitimate interests or obligations. Applicants must permit the FNR to share application data for evaluation, management, and statistical purposes. Compliance with the EU’s General Data Protection Regulation (GDPR) is mandatory for all FNR-funded projects. The FNR is not liable for any consequences arising from the misuse of login details or fraudulent use of confidential data. Data sharing is encouraged to prevent information isolation and promote knowledge reuse, with projects like DEA highlighting the benefits of such practices.

The FNR (Fonds National de la Recherche) emphasizes the importance of adhering to ethical principles and research integrity in all FNR-funded research activities, both inside and outside academia. Research proposals submitted to the CORE programme must comply with the FNR Research Integrity Guidelines and the ethical standards outlined in the Charter of Fundamental Rights of the European Union. Any research misconduct, such as non-compliance with ethical regulations, provision of false information, plagiarism, or data falsification, can lead to proposal rejection and further actions by the FNR, including grant suspension or termination, reimbursement requests, and legal actions.

The FNR also values the impact of research outputs on science and society, encouraging contributions beyond traditional research articles, such as data, software, mentoring, and policy changes. The merit review process for evaluating proposals is designed to be transparent, impartial, confidential, and adhere to international standards, as defined by the 2012 Global Summit on Merit Review. Panel experts are required to read the Programme Description, FNR Ethics Charter, and Code of Conduct for Research Assessment, and sign a Participation Form before reviewing proposals.

Beneficiaries of FNR funding must ensure compliance with the FNR research integrity guidelines and ethical charter, available on the FNR website. Host institutions are responsible for obtaining necessary authorizations from ethical and data protection committees. The FNR endorses the European Code of Conduct for Research Integrity and the Singapore Statement on Research Integrity. Additionally, all publications and electronic materials should acknowledge FNR funding by including the FNR logo and mentioning the support provided by the Fonds National de la Recherche, Luxembourg.

\section{Comment on Resubmission (if applicable)}

The evaluation of research proposals should focus on the quality and impact of scientific outputs, rather than relying on journal-based metrics. This approach aligns with the principles of the Declaration on Research Assessment (DORA), which the FNR has adopted. Reviewers are encouraged to consider all types of research outputs, including preprints, publications, data, software, and intellectual property, as well as the training of young scientists. The FNR's selection process involves an administrative eligibility check and emphasizes the importance of disseminating research to the public and media to generate broader societal impact.

In the context of model performance, SpaceRoBERTa ranks highest on several labels, showing significant improvements over the baseline BERT model, particularly in GN&C, Space environment, Thermal, and Structure & mechanics. The Nemenyi post-hoc test is used to compare model performances, highlighting differences in ranking. The evaluation of these models, as shown in Table 4.12, underscores the importance of focusing on content quality and the diverse range of research outputs, which vary across disciplines and individuals.

The evaluation process for funding proposals involves a comprehensive review and discussion by a panel of experts. Each proposal is initially reviewed by two experts who are closest to the domain of the proposal, although the panel members are generalists and not necessarily experts in the specific field. The panel meeting begins with the presentation of a synthesis of the written evaluations by the assigned panel expert, focusing on the proposal's objectives, fulfillment of selection criteria, and any conflicting statements from reviewers. The strengths and weaknesses of the proposal are highlighted, and an overall assessment is provided, including necessary modifications if applicable.

Panel members are encouraged to engage in discussions, raise concerns, and ask questions, regardless of their field of expertise. Ethical considerations are addressed if the proposal raises any issues. The panel chair invites members to discuss the evaluation findings and adjust the proposal's rating if necessary, providing justification for any changes. A consensus on funding recommendations is sought, and if achieved, a vote is conducted, requiring a two-thirds majority for a positive recommendation.

The panel must declare any conflicts of interest, and proposals not meeting minimal quality criteria are not discussed further, though their low scoring is verified. The final evaluation and recommendation are submitted to the FNR, and the 'Panel Conclusion' is the only feedback sent to the applicant. The funding decision is communicated to both the applicant and their supervisor(s). Additionally, a feedback loop is recommended to allow continuous improvement and learning, with users able to comment on outputs and provide ratings, enhancing the tool's effectiveness.

The text collection for the study is built on four heterogeneous data sources: ESA feasibility reports generated during CDF studies, academic publications, books, and Wikipedia pages. These sources are peer-reviewed and contain content verified by humans, with CDF reports being a primary source of information as cited by experts during the 2018 survey. The study also involves the integration of AI in space mission design, with past and ongoing studies focusing on this integration. The research outputs, including preprints, publications, data, and software, are expected to demonstrate value and impact on industry, policymakers, and society. The FNR encourages the dissemination of research to the general public and media, emphasizing the need for impact-generating activities from the initial project planning stage. The study employs a combination of Natural Language Processing (NLP) and Knowledge Graphs (KG) for heritage analysis, with applications such as rule-based inference for automatic mass budget generation. The research findings and contributions are evaluated, with limitations acknowledged and future research directions proposed. The data and findings are openly available from the University of Strathclyde KnowledgeBase.

The FNR has established a comprehensive and transparent process for evaluating research proposals, adhering to international standards of merit review. This process begins with an administrative eligibility check to ensure compliance with formal requirements. Proposals deemed eligible are then allocated to panel members and ad-hoc peer reviewers for remote evaluation. Panel experts are required to familiarize themselves with the Programme Description, the FNR Ethics Charter, the Code of Conduct for Research Assessment, and the Peer Review Guidelines. They must also sign a Participation Form and thoroughly review their assigned proposals.

During the panel meeting, members must declare any conflicts of interest. The panel expert assigned to each proposal presents a synthesis of the written evaluations, focusing on the proposal's objectives, fulfillment of selection criteria, strengths and weaknesses, overall assessment, and any necessary modifications. Following this presentation, the Panel Chair invites members to discuss the findings and adjust the proposal's rating if necessary.

The panel finalizes a 'Panel Conclusion' for each proposal, which serves as the basis for the formal decision process within the FNR. This conclusion is the only feedback sent to the applicant, while the funding decision is communicated to both the applicant and their supervisor(s). The FNR, as a signatory of the DORA declaration, evaluates research quality and impact independently of journal-based metrics, valuing a range of research outputs. The FNR may request supplementary information from the Beneficiary or Host Institution at any time, and failure to comply may result in grant termination or refund requests. The FNR is committed to maintaining transparency, impartiality, confidentiality, and integrity throughout the merit review process.

The FNR (Fonds National de la Recherche) emphasizes the importance of research impact on industry, policy makers, and society. Applicants are expected to list the value and impact of all research outputs, including preprints, publications, data, and training of young scientists. The FNR encourages dissemination of research to the public and media, and activities aimed at generating impact should be planned from the project's inception. Beneficiaries must maintain a presence in their departments and adhere to the project's work plan. The FNR, committed to the European Charter for Researchers, has signed the DORA Declaration, assessing research quality and impact independently of journal-based metrics. The AFR review process involves an international expert panel that evaluates proposals based on eligibility, ethical considerations, and selection criteria, focusing on the content and quality of scientific outputs. Proposals are rated from excellent to fair/poor, with strengths, weaknesses, and necessary modifications noted. High-quality publications from FNR-funded research are expected to be Open Access, aligning with FNR's Open Access Policy. The expert panel, nominated annually, analyzes proposals, rates them, and issues funding recommendations, ensuring a comprehensive evaluation process.

\section{Bibliography}

The FNR places significant emphasis on the impact of research outputs on science, industry, policy making, and society at large. To maximize this impact, FNR-funded research results are expected to be disseminated through high-quality, Open Access publications, in line with the FNR Policy on Open Access. This includes the encouragement of depositing preprints in open access repositories. The FNR also supports the dissemination of research to the general public and media, ensuring that activities aimed at generating impact are integrated from the initial project planning stages. Costs for project-related publications can be refunded through the FNR’s Open Access Fund.

The importance of research outputs extends beyond traditional publications, encompassing a diverse range of outputs such as data, reagents, software, and the training of skilled young scientists. The focus is on the content and quality of these outputs rather than their quantity or the metrics of the journals in which they are published. This approach recognizes the varied nature of important outputs across different disciplines and individuals.

Collaborative efforts, such as those facilitated by SnT’s Partnership Programme, further enhance the impact of research by addressing key challenges in industry and the public sector, particularly in ICT. This dynamic interdisciplinary environment has led to significant developments, including the launch of numerous EU and ESA projects, the protection and licensing of intellectual property, and the creation of spin-offs.

In addition to academic publications, other sources of information, such as ESA feasibility reports, books, and open-source platforms like Wikipedia, are valuable for disseminating knowledge. These sources are peer-reviewed and verified, ensuring the reliability of the information provided. The integration of these diverse sources supports the comprehensive dissemination of research findings, contributing to the broader impact on society.

The FNR (Fonds National de la Recherche) has implemented a comprehensive approach to evaluating research proposals, emphasizing the quality and impact of research outputs over traditional journal-based metrics. As a signatory of the Declaration on Research Assessment (DORA), the FNR encourages applicants to highlight a diverse range of research outputs, such as datasets, software, intellectual property, and the training of young scientists, rather than relying on journal impact factors. This approach aligns with the FNR's commitment to assess the scientific content and potential impact of research independently of publication metrics.

The selection process for AFR grants involves several stages, beginning with an administrative eligibility check. Proposals are then rated based on criteria such as the project's scientific quality, originality, clarity of objectives and methods, feasibility, and the applicant's profile and potential. Ethical considerations are also reviewed, and any research misconduct, such as plagiarism, may lead to proposal rejection.

The FNR values the dissemination of research to the general public and media, encouraging activities that generate societal impact from the initial project planning stage. The organization supports a merit review process that adheres to international standards, ensuring that the most important strengths and weaknesses of each proposal are thoroughly assessed. Overall, the FNR's evaluation framework prioritizes the content and quality of scientific outputs, recognizing the diverse contributions that extend beyond traditional research articles.

The selection criteria for AFR individual grants focus on the project's scientific quality and the applicant's profile. The project should be excellent, original, well-articulated, and feasible. The applicant's curriculum vitae, achievements, and degree levels are also considered. The FNR, which has signed the Declaration on Research Assessment (DORA), evaluates proposals based on quality and impact rather than journal-based metrics, valuing all research outputs. The selection process includes an administrative eligibility check and adheres to international merit review standards. Research integrity is crucial, with misconduct such as false information or plagiarism leading to proposal rejection. An automated plagiarism check is conducted on randomly chosen applications. The FNR's commitment to a fair review process is reflected in its guidelines for selection panels. Additionally, the selection of journals for research is based on criteria such as being peer-reviewed, having high citation scores, and covering a broad range of topics. Only recent articles from 2017 to 2019 are considered to ensure the inclusion of the latest research findings.

The selection of journals for research publication was based on specific criteria, including being peer-reviewed with high citation scores, covering a broad range of topics, and being accessible either freely or through a university subscription. Only articles published between 2017 and 2019 were considered to ensure the inclusion of the most recent research findings. The focus is on the content and quality of scientific outputs rather than the number of publications, the venue, or aggregate metrics. This approach acknowledges the diverse range of research-related and non-research-related outputs, which can include data, reagents, software, mentoring, and intellectual property, among others.

The FNR emphasizes the importance of the impact of research outputs on science, industry, policy-making, and society. To maximize this impact, FNR-funded research results are expected to be disseminated through high-quality, Open Access publications, with costs potentially covered by the FNR’s Open Access Fund. Researchers are encouraged to list the value and impact of all research outputs, including preprints, publications, data, and training of young scientists. The dissemination of research to the general public and media is also encouraged, with activities aimed at generating impact being planned from the project's inception.

Through SnT’s Partnership Programme, researchers collaborate with over 70 private and public organizations to address key challenges in ICT, contributing to a dynamic interdisciplinary research environment. The Centre has rapidly developed since its launch in 2009, engaging in over 100 EU and ESA projects, protecting and licensing IP, launching spin-offs, and fostering a community of around 480 people. The scientific content of research articles is prioritized over publication metrics or the journal's identity, recognizing legitimate delays in research activity due to personal factors.

The selection of journals for research publication is based on specific criteria, including being peer-reviewed with high citation scores, covering a broad range of topics, and being accessible either freely or through a university subscription. Only articles published between 2017 and 2019 were considered to ensure the inclusion of the most recent research findings. The focus is on the content and quality of scientific outputs rather than the number of publications, the venue, or aggregate metrics. The FNR, as a signatory of the DORA declaration, emphasizes the importance of listing a diverse range of research outputs, such as datasets, software, training, and intellectual property, rather than relying on journal-based metrics like Journal Impact Factors. The FNR also highlights the significance of the impact of research outputs on science, industry, policy making, and society. To maximize this impact, FNR-funded research results are expected to be disseminated through high-quality, Open Access publications, with costs potentially refunded through the FNR’s Open Access Fund. Additionally, the FNR encourages the deposition of preprints in open access repositories and the dissemination of research to the general public and media. Activities aimed at generating impact should be planned from the project's inception, taking into account legitimate delays in research activity due to personal factors.

\end{document}